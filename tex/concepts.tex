This chapter introduces the fundamental concepts related to the developments presented in this thesis. The major areas introduced are: Epigenetics, Genetics, (Epi)Genomic data analysis.


\section{Epigenetics alterations}
\subsection{DNA methylation}
\section{Histone modifications}

\section{Genetics alterations}


\section{(Epi)Genomic data analysis}

\subsection{Statistical analysis}

\subsubsection{Hypothesis Testing}

Human genetic studies aims to identify if a phenotype is related to the genotypes
at various loci, that is, if genetic variations have an influence on risk of
disease or other health-related phenotypes.
Statistical analysis is a crucial to present the findings in an
interpretable and objective manner \cite{sham2014statistical}.

The most popular hypothesis testing approach used to test if
genotypes and phenotypes are related  is the frequentist significance testing approach.
This is a classical approach that involves setting up two competing hypothesis: a
null hypothesis ($H_0$\simbolo{H_0}{Null hypothesis}) and an alternative hypothesis ($H_1$\simbolo{H_1}{Alternative hypothesis}).

Computing the statistical significance can be done using a one-tailed or a two-tailed test.
A two-sided test is appropriate to evaluate both direction of the test, for example,
is the estimated value smaller or higher than the reference, which acctualy test if the
 estimated value is different from the reference.
A one-sided test is  is appropriate to evaluate only one direction of the test,
for example, is the estimated v alue smaller than the reference.
An example in genetic studies for a two-sided test would be $H_0$
hypothesis that genotypes has no effect on the phenotypes
while the $H_1$ hypothesis is that there is a effect.
Table \ref{hypothesis-tests} shows other examples.

% Please add the following required packages to your document preamble:
% \usepackage{booktabs}
\begin{table}[h!]
\centering
\caption[Hypothesis tests]{Example of three hypothesis tests about the population mean $\mu$. In genetics it could be the mean level of expression of a gene.}
\label{hypothesis-tests}
\begin{tabular}{@{}lll@{}}
\toprule
\multicolumn{1}{c}{\textbf{Type}} & \multicolumn{1}{c}{\textbf{Null}} & \multicolumn{1}{c}{\textbf{Alternative}} \\ \midrule
Right-tailed & $H_{0}:\mu = 0$ & $H_{1}: \mu >  0 $  \\
Left-tailed & $H_{0}:\mu = 0$ & $H_{1}: \mu <  0 $  \\
Two-tailed & $H_{0}:\mu = 0$ & $H_{1}: \mu \neq 0 $ \\ \bottomrule
\end{tabular}
\end{table}

\subsubsection{Making a decision: P-value approach}

The decision to reject or accept $H_0$ is made based on the calculation of a test statistic (T) from the observed data.
As the value of T depends on particular individuals in the population, repeating the study
using  different random samples from the population would provide of many different values for T.
These set of T can be summarized as a probability distribution.

% Errors in hypothesis testing
Even though, the decision made to reject or accept $H_0$ just state that we had
enough evidence to behave one way or the other.
The rejection of the null hypothesis does not prove that the alternative hypothesis is true as
the acceptance the null hypothesis does not prove that the null hypothesis is true.
It might happen that null hypothesis was reject when it was true, or it was not
rejected when it was false. The first error in statistics is called a Type I error ("false positive"),
 while the second is called a Type II error ("false negative").
Table \ref{type_errors} shows the relations between truth/falseness of the null hypothesis and outcomes of the test.

It is denotated  rate of the type I error  or significance level
$\alpha$\simbolo{\alpha}{Significance level} the probability of having a false positive.
Normally, the significance level is set to 5\%, implying that it is acceptable to have a 5\%
probability of incorrectly rejecting the null hypothesis. With the same logic, the rate of the
type II error is denoted by $\beta$\simbolo{\beta}{rate of the type II error}.

% Please add the following required packages to your document preamble:
\begin{table}[]
  \centering
  \caption{Type I and II Errors. $\alpha = P(\textrm{Type I Error)}$, $\beta = P(\textrm{Type II Error})$}
  \label{type_errors}
  \begin{tabular}{ccc}
    \toprule
    \textbf{Decision} & \textbf{$H_0$ is True} & \textbf{$H_0$ is False} \\ \midrule
  Do Not Reject $H_0$ & Correct Decision  & Incorrect Decision (1 - $\beta$)\\
  Rejct $H_0$ & Incorrect Decision (1 -  $\alpha$)& Correct Decision \\ \bottomrule
  \end{tabular}
\end{table}

To make a decision wheter to reject or accept the null hypothesis, the concept of
probability value was introduced.
\citeonline{wasserstein2016asa} defined a p-value  ($\textrm{p-value}\in [0,1]$) as the probability under a specified statistical model, constructed under a set of assumptions (normally “null hypothesis"), that a statistical summary of the data
(e.g., the sample mean difference between two compared groups) would be equal to or more extreme than its observed value \cite{wasserstein2016asa}. That means, the smaller the p-value, the greater the statistical incompatibility of the data with the null hypothesis and greater the p-value more compatibible is the data with the null hypothesis.
In summary, if P-value is small (e.g.$\textrm{P-value} \leq \alpha$) then the null hypothesis is rejected,
otherwise it is not rejected.

It is important to highlight that a p-value does not measure the size of an effect or the importance of a result.
It might happen that a very small effect produces smaller p-values if the
sample size is big or measurement precision is high.
On the other hand, a large effect might produce higher p-values if
the sample size is small or measurements are imprecise.


%The P-value approach consists in the following steps to conducting any hypothesis test:
%\begin{enumerate}
%  \item Set $H_0$ (null hypotheses) and $H_1$ (alternative hypotheses)
%  \item Using the sample data and assuming the null hypothesis is true, calculate the value of the test statistic.
%  Again, to conduct the hypothesis test for the population mean $\mu$, we use the t-statistic $t^{\ast}= \frac{\bar{x}-\mu}{s/\sqrt{n}}$ which follows a t-distribution with n - 1 degrees of freedom.
%  \item Using the known distribution of the test statistic, calculate the P-value: "If the null hypothesis is true, what is the probability that we'd observe a more extreme test statistic in the direction of the alternative hypothesis than we did?"
%  \item Set the significance level, $\alpha$, the probability of making a Type I
%  error to be small - 0.01, 0.05, or 0.10. Compare the P-value to  $\alpha$.
%  If the P-value is less than (or equal to)  $\alpha$, reject the null hypothesis
%  in favor of the alternative hypothesis. If the P-value is greater than  $\alpha$,
%  do not reject the null hypothesis.
%\end{enumerate}


%For a one-sided test (for example, a test for effect size greater than zero), the definition of the P value is slightly more complicated: P* = P/2 if the observed effect
%is in the pre-specified direction, or P* = (1 – P)/2 otherwise, where P is defined as above. In the Neyman–Pearson hypothesis testing framework, if the P value is smaller than a preset threshold α (for example, 5 × 10−8 for genome-wide association studies), then H is rejected and the result is considered to be significant.

%By setting up a hypothesis test in this manner, the probability of making the error of
% rejecting H0 when it is true (that is, a type 1 error) is ensured to be α. However, another possible type of error is the failure to reject H0 when it is false (that is, type 2 error, the probability of which is denoted as β). Statistical power is defined as 1 – β (that is,
%the probability of correctly rejecting H0 when a true association is present).


\subsubsection{Correcting for multiple testing}

When performing  a set of statistical inferences simultaneously more likely erroneous inferences are to occur.
For example, if 100 tests are carried
out, then 5\% of them (that is 5 tests) are expected to
have $P-value < 0.05$ by chance when $H_0$ is in fact true for all the tests.
Compared to a single test (equations \ref{eq_error} and \ref{eq_error2}), the probability of having a type 1 error multiple test is given by the equations \ref{eq_multiple_error} and \ref{eq_multiple_errorb} \cite{vsidak1967rectangular}.

\begin{subequations}

\begin{align}
  P(\textrm{Making an error)} = \alpha \label{eq_error}\\
  P(\textrm{Not making an error)} = 1 - \alpha \label{eq_error2}\\
  P(\textrm{Not making an error in m tests)} = (1 - \alpha)^m  \label{eq_multiple_error}\\
  P(\textrm{Making at least 1 error in m tests}) = 1 - (1 - \alpha)^m  \label{eq_multiple_errorb}
\end{align}
\end{subequations}

To handle this multiple statistical testing problem,
some techniques  to re-calculating probabilities obtained from a statistical test which was repeated multiple times
have been developed to prevent the inflation of false positive rates.


Among the different approaches to control type I errors we have
\sigla{FWER}{Family-wise error rate} which controls the probability of at least one type I error,
and \sigla{FDR}{False discovery rate}
which controls the expected proportion of Type I errors
among the rejected hypotheses. Compapred to FDR, controlling FWER is extremely conservative
approach as the power to detect $H_1$ gets very small.

Among the different adjustment methods to control FWER includes the Bonferroni correction
  in which the p-values are multiplied by the number of comparisons ($M * P_i < \alpha$) and the Holm correction, $P-adjusted_i = P_i * (M + 1 - i)$, where $i \in \{1,2,\dots,n\}$ and smaller the p-value is smaller will the index $i$ be \cite{aickin1996adjusting}.
 The Benjamini-Hochberg (BH) method to control FDR procedure will identify the largest $k$,
 such that $P_k \leq \frac{k}{m}\alpha$, all null hypotheses $H_i$ for $i \in \{1,\ldots,k\}$ are rejected.


These methods makes the assumption that
the tests are independent tests, which often is not valid for genomics data.
For dependent tests, permutation methods are often used to calculate
 p-values.  This approach recalculate a p-value comparing the P-value calculated
 from the real data test with random ones,
 which are performed by randomly shuffling the case–control (or phenotype)
 labels. All $M$ tests are recalculated on the reshuffled data set, with the smallest P value of these M tests being recorded. The procedure is repeated for many times to construct an empirical frequency distribution of the smallest P values.
This  empirical adjusted P value ($P_{*}$) is given by: $$P_{*} = \frac{r + 1}{n + 1}$$ where $n$ are the number of
permutation carried out, and $r$ is the number of permutated p-values smaller than P-value calculated
from the real data.

For example, considering $\textrm{P-value = 0.1}$ and the
permutated p-values $$P_{permu} =\{0.001,0.01,0.02,0.03,0.05,0.2,0.5,0.6,1\}$$ the first 5 permutated p-values
are smaller than the original p-value, which would give us $r = 5$, resulting in:
$$P_{*} = \frac{r + 1}{n + 1} =  \frac{5 + 1}{9 + 1} = 0.6 $$
It is important to highlight that
a high number of permutations is required in order to produce reliable permutated p-value adjusted.
 \cite{davison1997bootstrap,north2002note,north2003note,sham2014statistical}.


\subsubsection{Nonparametric and parametric tests}

Statistical procedures can be classified into two groups:  Parametric and nonparametric.
Parametric statistical procedures rely on assumptions about the shape of the distribution
(i.e., assume a normal distribution) in the underlying population and about the form or
parameters (i.e., means and standard deviations) of the assumed distribution.
While nonparametric statistical procedures doest not rely or rely on only few assumptions about the shape or
parameters of the population distribution from which the sample was drawn.
Some of these producedures are summarized in Table \ref{Parametric-nonparametric}.

The t-test, a parametric test, is the most widely used statistical test for comparing the means of two independent groups.
It assumes that the data are distributed Normally, that samples from different groups are independent and that the variances between the groups are equal. The most commonly used nonparametric test in this situation is the \sigla{WRST}{Wilcoxon Rank Sum Test} and the closely related \sigla{MWU}{Mann-Whitney U-test}. The WRST assumes that observations from the different groups are random samples (i.e. independent and identically distributed) from their respective populations and are mutually independent and that the observations are ordinal or continuous measurements.
When there are more than two groups being compared,
the nonparametric test used is \sigla{KW}{Kruskal-Wallis test}, a generalization of the WRST. KW is the nonparametric equivalent to \sigla{ANOVA}{Analysis of variance}.


\bgroup
\def\arraystretch{2.0}%  1 is the default, change whatever you need

\begin{table}[h!]
\footnotesize
\centering
\caption{Summary of parametric and non parametric procedures.}
\label{Parametric-nonparametric}
\begin{tabular}{p{3.5cm}p{4cm}p{3cm}p{3cm}}
\toprule
\textbf{Analysis Type} & \textbf{Example} & \textbf{Parametric} & \textbf{Nonparametric} \\ \midrule
Compare means between two distinct/independent groups & Is the mean TP53 gene expression for control group different from the mean for treatment group? & Two-sample t-test & Wilcoxon ranksum test \\
Compare two quantitative measurements taken from the same individual & Was there a change in gene expression after the treatment? & Paired t-test & Wilcoxon signedrank test \\
Compare means between three or more distinct/independent groups & For a given three groups (e.g., placebo, drug \#1, drug \#2), is the TP53 gene expression different among the three groups? & Analysis of variance (ANOVA) & Kruskal-Wallis test \\
Estimate the degree of association between two quantitative variables & Is age related to the TP53 gene expression? & Pearson coefficient of correlation & Spearman’s rank correlation \\ \bottomrule
\end{tabular}
\end{table}
\egroup

% https://www.ncbi.nlm.nih.gov/pmc/articles/PMC2743502/
% http://blog.minitab.com/blog/adventures-in-statistics-2/choosing-between-a-nonparametric-test-and-a-parametric-test


\subsection{Machine Learning}
