
\newcommand{\comando}[1]{\textbf{$\backslash$#1}}


\section{Objectives}

\section{Motivations}

\section{Organization of the Remainder of the Document}

This paper is organized in chapters as follows. In Chapter 2, we introduce
the fundamental concepts required to the development 
of this project. With this respect, we review the basic definitions genetics and epigenetics area. After presenting the theoretical concepts required to understand the work, we present the results of the work in Chapters 3, 4, and 5. Specifically, in
Chapter 3, we explore the methods and computational tools developed. 
In Chapter 4, we show the results of the use of the methods and tools presented in chapter 3 another real-world application. Enclosing this thesis, in Chapter 5, we draw the
main conclusions of this work. Moreover, the scientific contributions derived from this
project are listed, as well as possible future works. Lastly, we show the main papers
that have been published during the Doctorate period.

%Este documento explica brevemente como trabalhar com a classe \LaTeX~\textit{icmc} para confeccionar trabalhos acadêmicos seguindo as normas da \sigla{ABNT}{Associação Brasileira de Normas Técnicas} e as \aspas{\textit{Diretrizes para apresentação de dissertações e teses da USP: documento eletrônico e impresso. Parte I (ABNT)}}, publicado pelo \sigla{SIBi}{Sistema Integrado de Bibliotecas} USP. O presente manual também atende as exigências prevista no regimento do Programa de Pós-graduação em \sigla{CCMC}{Ciências da Computação e Matemática Computacional} do \sigla{ICMC}{Instituto de Ciências Matemáticas e de Computação} da \sigla{USP}{Universidade de São Paulo}.
