
\section{Sotwares developed}

\section{TCGAbiolinks: An R/Bioconductor package to download and analyze data from GDC}

The aim of TCGAbiolinks is four-fold: (i) to facilitate
data retrieval via GDC’s API; (ii) to prepare the data
using the appropriate preprocessing strategies; (iii) to provide
a means to conduct different standard analyses and
advanced integrative analyses and (iv) to allow the user to
easily reproduce earlier research results.
We introduce public methods
used in several marker papers to integrate DNA methylation
and gene expression data. In addition, our tool extracts
published molecular subtype information for each
TCGA sample within a tumor type (generally embedded in
supplementary tables, PDFs or external websites). The tool was developed in the  R language specifically
for integration within the Bioconductor project, thus we have provided most of the data objects as the Bioconductor specified
‘SummarizedExperiment’ class \cite{huber2015orchestrating},
thereby allowing easy integration with other data types and statistical
methods that are common in the Bioconductor repository.

\subsection{The TCGAbiolinks package}
TCGAbiolinks is an R package, which is licensed under
the General Public License (GPLv3), and is freely available
through the Bioconductor repository \cite{gentleman2004bioconductor}. By conforming to the strict guidelines for package submission to
Bioconductor, we were able to utilize and incorporate existing R/Bioconductor packages and statistics to assist in identifying differentially altered genomic regions defined by mutation, copy number, expression or DNA methylation; to reproduce previous TCGA marker studies; and to integrate data types both within TCGA and across other data types outside of TCGA. TCGAbiolinks consists of functions that can be grouped into three main levels: Data, Analysis and Visualization. More specifically, the package provides multiple methods for the analysis of individual experimental platforms (e.g. differential expression analysis or identifying
differentially methylated regions or copy number alterations) and methods for visualization (e.g. survival plots, volcano plots and starburst plots) to facilitate the development of complete analysis pipelines. In addition, TCGAbiolinks
offers in-depth integrative analysis of multiple platforms,
such as copy number and expression or expression
and DNA methylation, as demonstrated and applied in our
recent TCGA study of 1122 gliomas \cite{ceccarelli2016molecular}. These functions
can be used independently or in combination to provide the
user with fully comprehensible analysis pipelines applied to
TCGA data. A schematic overview of the package is presented
in Figure \ref{fig:tcgabiolinksfunctions}. The next subsections describe each of the three main levels (Data, Analysis and Visualization), highlighting the importance and utility of each associated function and sub-function.

\begin{figure}
\centering
\includegraphics[width=1.0\linewidth]{images/figure1_new.pdf}
\caption[Overview of TCGAbiolinks functions.]{ Overview of TCGAbiolinks functions. TCGAbiolinks is organized in three categories. In the first category (Data), functions to query the GDC database, to download the data and to prepare it are made available. The second category (Analysis) contains functions that allow the user to carry out different types of analyses; these include clustering (\textit{TCGAanalyze\_Clustering}), differential expression analysis (\textit{TCGAanalyze\_DEA}) and enrichment analysis (\textit{TCGAanalyze\_EA}). Finally, the obtained results can be visualized using the functions in the third category (Visualization): these include principal component analysis (\textit{TCGAvisualize\_PCA}), starburst plots (\textit{TCGAvisualize\_starburst}) and survival curves (\textit{TCGAvisualize\_SurvivalCoxNET}). The different dependencies to other R/Bioconductor packages are specified in the last row of the figure.  }
\label{fig:tcgabiolinksfunctions}
\end{figure}

\subsubsection*{Data}

TCGA data is accessible via the \href{https://gdc-portal.nci.nih.gov/}{NCI Genomic Data Commons (GDC) data portal}, \href{https://gdc-portal.nci.nih.gov/legacy-archive/search/f}{GDC Legacy Archive} and \href{https://github.com/BioinformaticsFMRP/TCGAWorkflow/blob/master/vignettes/gdac.broadinstitute.org}{the Broad Institute’s GDAC Firehose}. The GDC Data Portal provides access to the subset of TCGA data that has been harmonized against \sigla{GRCh38}{Genome Reference Consortium Human Build 38} (hg38) using GDC Bioinformatics Pipelines which provides methods to the standardization of biospecimen and clinical data, the re-alignment of DNA and RNA sequence data against a common reference genome build GRCh38, and the generation of derived data. Whereas the GDC Legacy Archive provides access to an unmodified copy of data that was previously stored in CGHub \cite{wilks2014cancer} and in the TCGA Data Portal hosted by the TCGA Data Coordinating Center (DCC), in which uses as references \sigla{GRCh37}{Genome Reference Consortium Human Build 37} (hg19) and \sigla{GRCh36}{Genome Reference Consortium Human Build 36} (hg18).

The previously stored data in CGHub, TCGA Data Portal and Broad Institute’s GDAC Firehose, were provided as different levels or tiers that were defined in terms of a specific combination of both processing level (raw, normalized, integrated) and access level (controlled or open access). Level 1 indicated raw and controlled data, level 2 indicated processed and controlled data, level 3 indicated Segmented or Interpreted Data and open access and level 4 indicated region of interest and open access data. While the TCGA data portal provided level 1 to 3 data, Firehose only provides level 3 and 4. An explanation of the different levels can be found at TCGA Wiki (\burl{https://wiki.nci.nih.gov/display/TCGA/Data+level}). However, the GDC data portal no longer uses this based classification model in levels. Instead a new data model was created, its documentation can be found in GDC documentation at \burl{https://gdc.nci.nih.gov/developers/gdc-data-model/gdc-data-model-components}. In this new model, data can be open or controlled access. While the GDC open access data does not require authentication or authorization to access it and generally includes high level genomic data that is not individually identifiable, as well as most clinical and all biospecimen data elements, the GDC controlled access data requires dbGaP authorization and eRA Commons authentication and generally includes individually identifiable data such as low level genomic sequencing data, germline variants, SNP6 genotype data, and certain clinical data elements. The process to obtain access to controlled data is found in GDC web site at \burl{https://gdc.nci.nih.gov/access-data/obtaining-access-controlled-data}.

TCGAbiolinks divides the GDC data retrieval into three main functions: \textit{GDCquery}, \textit{GDCdownload}
and \textit{GDCprepare}.
\textit{GDCquery} allows the user to query data from the the The NCI's Genomic Data Commons (GDC) data portal or GDC Legacy Archive by accessing the database API. Up to the moment, the GDC data portal provides data from two programs The Cancer Genome Atlas (TCGA) and Therapeutically Applicable Research to Generate Effective Treatments (TARGET)  on more
than 38 diseases type cancer types and 5 different molecular data types (Transcriptome Profiling, Raw Sequencing Data, Copy Number Variation, DNA Methylation) as well as 3 different types of clinical reports (clinical tables, pathology reports and histology image slides). The pathology
reports and histology image slides are not prepared
but are downloaded from the GDC Legacy Archive to a directory if requested by the user. \textit{GDCdownload} receives the output of \textit{GDCquery}, a file manifest with all the metadata necessary to download it. In order to organize the files downloaded, this function will save the data with the following pattern "Root directory/project/source/data\_category/data\_type/file\_id/file\_name" (i.e Example: GDCdata/TCGA-GBM/harmonized/DNA\_Methylation/Methylation\_Beta\_Value/079fcaff-3ae6-4150-b2e6-2b7330ffbcd9/jhu-usc.edu\_GBM.HumanMethylation450.10.lvl-3.TCGA-19-A6J5-01A-21D-A33U-05.gdc\_hg38.txt)
If a file was
previously downloaded it will not be re-downloaded. \textit{GDCprepare} is a function that reads open processed data and prepares them for downstream analysis. Specifically, the objects are organized in a \textit{SummarizedExperiment}
object to allow easy integration
with other Bioconductor packages, such as GRanges \cite{lawrence2013software},
IRanges \cite{lawrence2013software}, limma \cite{ritchie2015limma} and edgeR \cite{robinson2010edger}. The samples are
always referred to by their given TCGA/TARGET barcode. If the user prefers the data not to be prepared in a \textit{SummarizedExperiment},
there is an option to set the argument \textit{SummarizedExperiment}
to FALSE; the data are then prepared as
a standard data frame object (rows and columns).

\subsubsection*{Analysis}
The analysis functions and subfunctions are designed to analyze TCGA data through both common and novel methods.
The main function, called \textit{TCGAanalyze}, comprises two distinct types of analysis: molecular analysis and clinical analysis. Once the data are prepared into data matrices
(genes/loci in rows and samples in columns) or a SummarizedExperiment, the downstream
analysis can be divided into (i) supervised analysis: differential expression analysis, enrichment analysis and master regulator analysis or (ii) unsupervised analysis: inference of gene regulatory network, clustering, classification, \sigla{ROC}{Receiver Operator Characteristics} \cite{sonego2008roc}, \sigla{AUC}{Area Under the Curve}, feature selection and survival analysis.
\textit{TCGAanalyze\_Normalization} allows users to normalize mRNA transcripts and miRNA using the EDASeq package \cite{risso2011gc}. This function uses within-lane normalization procedures to adjust for GC-content effects (or other gene-level effects) on read counts: LOESS robust local regression and global-scaling, full-quantile and between-lane normalization procedures to adjust for distributional differences between lanes (e.g. sequencing depth).
\textit{TCGAanalyze\_DEA} allows the user to identify differential expression or regions between two populations or conditions.
In particular, we used the edgeR package from
Bioconductor, which uses the \sigla{qCML}{quantile-adjusted conditional
maximum likelihood} method for experiments with
a single factor to detect \sigla{DEGs}{differentially expressed genes} \cite{robinson2010edger}. Compared to several other estimators, qCML is the most reliable in terms of bias on a wide range of conditions; specifically, qCML performs best in situations involving many small samples with a common dispersion \cite{robinson2007small}. The P-values generated from the analysis are sorted in ascending order and corrected using the Benjamini \& Hochberg procedure for multiple testing correction \cite{Ben95}.
After running \textit{TCGAanalyze\_DEA}, it is possible to filter the output by fold change and/or significance and to use the \textit{TCGAanalyze\_LevelTab} function to create a table of DEGs, including \sigla{FC}{fold change}, \sigla{FDR}{false discovery rate}, gene expression levels of samples under conditions of interest and delta values (the difference in gene expression multiplied by logFC).
\textit{TCGAanalyse\_DMR} allows the user to identify differentially methylated regions (DMRs) between two groups with a DNA methylation difference above a certain threshold. To calculate P-values, this subfunction uses the Wilcoxon ranksum statistical non-parametric test and adjusts the values using the FDR method.
\textit{TCGAanalyze\_Clustering} allows the user to perform a hierarchical cluster analysis through two methods: ward.D2 and ConsensusClusterPlus \cite{wilkerson2010consensusclusterplus}.


\subsubsection*{Visualization}
The visualization section allows the user to visualize the results generated by the analysis sections using heatmap, cluster, plots with incremental layers (ggplot2), pathway enrichment analysis and PCA. Furthermore, we provide methods to generate a starburst plot, which  integrates gene expression and DNA methylation data \cite{noushmehr2010identification}.

\subsection{Comparisons}

 Recently, several tools to retrieve TCGA data sets have been made available, as summarized in Table \ref{tab:tcgabiolinks-fig1}. These tools
 include TCGA-Assembler \cite{zhu2014tcga}, CGDS-R \cite{gao2013integrative}, canEnvolve \cite{samur2013canevolve}, Firehose \cite{deng2017firebrowser}, RCTCGAtoolbox \cite{samur2014rtcgatoolbox}, and  cBioPortal \cite{cerami2012cbio} . These tools can
 be divided into three representative categories. The first category comprises tools mainly used to download cancer genomics data, such as TCGA-Assembler and CGDSR.
 The second category includes tools that focus mainly
 on data analysis and integration, such as canEnvolve. The third category comprises tools to download and analyze data, such as RTCGAToolbox, Firehose and cBioPortal.

 RTCGAToolbox is a tool that systematically accesses the Broad GDAC Firehose (\burl{https://gdac.broadinstitute.org/}) preprocessed data and performs basic analysis and visualization of an individual data type (expression, mutation or DNA methylation). Despite the existence of TCGA specific software packages, none of these tools perform the integrative analysis harnessing methodologies designed by TCGA \sigla{AWGs}{Analysis working groups}, such as identifying epigenetically silenced
 genes (represented in a starburst plot \cite{noushmehr2010identification} or functional
 copy number identification \cite{ceccarelli2016molecular}. Although RTCGAToolbox can download and analyze Firehose-generated data, neither tool can provide the downloaded data as a "SummarizedExperiment" object, which is critical for allowing the full integration and use of other popular Bioconductor packages, an integral aspect of Bioconductor \cite{gentleman2004bioconductor,gentleman2004bioconductor}. Briefly, the SummarizedExperiment class is a matrix-like container in which rows represent ranges of interest (as a GRanges or GRangesList object) and columns represent samples (with sample data summarized as a DataFrame). A Summarized Experiment contains one or more assays, each represented by a matrix-like object of numeric or other mode.
Finally, TCGAbiolinks is able to access data aligned against the Genome Reference Consortium Human Build 38 (hg38) via the
\href{https://gdc-portal.nci.nih.gov/}{NCI Genomic Data Commons (GDC) data portal}, and  data aligned against the Genome Reference Consortium Human Build 19 (hg19) via the \href{https://gdc-portal.nci.nih.gov/legacy-archive/search/f}{GDC Legacy Archive}. This feature is only available using TCGA-Assembler.


\bgroup
\def\arraystretch{1.5}%  1 is the default, change whatever you need
\begin{table}
\footnotesize
\centering
\caption[Comparing TCGAbiolinks to competing software]{Each column represents a software tool compared with TCGAbiolinks,
and each row represents a feature. The cells checked with $X$ indicates
features that exists in the tool. Available platform abbreviations are defined
as: R (R script); C (R package deposited in CRAN); B (Bioconductor
package); W (available only as a web portal);}
\label{tab:tcgabiolinks-fig1}
\begin{tabular}{p{3cm}p{5cm}|l|l|l|l|c|l|c|}
\cline{3-9}
 &  & \multicolumn{7}{c|}{\cellcolor[HTML]{333333}{\color[HTML]{FFFFFF} \textbf{Packages}}} \\ \cline{3-9}
 &  &  &  &  & & &  &  \\
 &  &  &  &  & & &  &  \\
 &  &  &  &  & & &  &  \\
 &  &  &  &  & & &  &  \\
 &  &  &  &  & & &  &  \\
{\cellcolor[HTML]{333333}{\color[HTML]{FFFFFF} \textbf{Features}}} & {\cellcolor[HTML]{333333}{\color[HTML]{FFFFFF} \textbf{Sub-features}}} & \multirow{-6}{*}{\rotatebox[origin=c]{90}{TCGAbiolinks}} &
\multirow{-6}{*}{\rotatebox[origin=c]{90}{TCGAAssembler}} &
\multirow{-6}{*}{\rotatebox[origin=c]{90}{canEnvolve}} &
\multirow{-6}{*}{\rotatebox[origin=c]{90}{TCGA2stat}} &
\multirow{-6}{*}{\rotatebox[origin=c]{90}{Firehose-FirebrowserR}} & \multirow{-6}{*}{\rotatebox[origin=c]{90}{RTCGAtoolbox}} &
\multirow{-6}{*}{\rotatebox[origin=c]{90}{cBio Portal CGDS-R}} \\ \hline
\multicolumn{1}{|l|}{Availability} & Platform & B & R & W & C & CW & B & CW \\ \hline
\multicolumn{1}{|l|}{} & Access to data aligned against the GRCh38/hg38 & X & X &  &  &  &  &  \\ \cline{2-9}
\multicolumn{1}{|l|}{\multirow{-2}{*}{Genome of reference}} & Access to data aligned against the GRCh37/hg19 & X & X & X & X & X & X & X \\ \hline
\multicolumn{1}{|l|}{Query TCGA Cases} & Individual TCGA samples (e.g. TCGA-01-0001) & X & X &  &  & X &  &  \\ \hline
\multicolumn{1}{|l|}{Download} & All TCGA platforms & X &  &  &  &  &  &  \\ \hline
\multicolumn{1}{|l|}{} & mRNA & X &  & X & X & X & X & X \\ \cline{2-9}
\multicolumn{1}{|l|}{} & miRNA & X &  & X & X & X & X & X \\ \cline{2-9}
\multicolumn{1}{|l|}{} & copy number & X &  & X & X & X & X & X \\ \cline{2-9}
\multicolumn{1}{|l|}{} & DNA methylation & X &  &  & X & X & X & X \\ \cline{2-9}
\multicolumn{1}{|l|}{} & Clinical & X &  & X & X & X & X & X \\ \cline{2-9}
\multicolumn{1}{|l|}{} & Protein &  &  & X &  & X &  & X \\ \cline{2-9}
\multicolumn{1}{|l|}{\multirow{-7}{*}{Data type analysis}} & Mutatation & X &  & X & X & X & X & X \\ \hline
\multicolumn{1}{|l|}{Integrative analysis} & DNA methylation and gene expression & X &  &  &  & X &  &  \\ \hline
\multicolumn{1}{|l|}{Other} & Extensible to other BioC packages & X &  &  &  &  &  &  \\ \hline
\end{tabular}
\end{table}
\egroup

\subsection{Software availability}

TCGAbiolinks is available under the \sigla{GNU GPL3}{GNU General Public License version 3}.
Its source code is available at \url{https://github.com/BioinformaticsFMRP/TCGAbiolinks}
 and a binary version for windows, macOSX and Linux is freely available through the Bioconductor repository at \burl{http://bioconductor.org/packages/TCGAbiolinks/}. To execute this tool, it is required to have installed a R version $\geq 3.3$.

\subsection{Public reception}

TCGAbiolinks had a good reception the research community being among the top 5\% of the most downloaded tools of the Bioconductor project. In October 2017, the tool already had more than 17 thousand downloads, with an average of visits to the documentation pages of a thousand users per month. The Figure \ref{fig:tcgabiolinksdownload} shows a summary since the beginning of the project.

\begin{figure}
	\centering
	\includegraphics[width=0.7\linewidth]{images/tcgabiolinks_download.pdf}
	\caption[TCGAbiolinks download summary]{TCGAbiolinks download summary. A) Number of downloads per month. B) Number of active users. C) Percentage of sessions connect to the documentation by country.}
	\label{fig:tcgabiolinksdownload}
\end{figure}






\section{TCGAbiolinksGUI: A graphical user interface to analyze GDC cancer molecular and clinical data}

The National Cancer Institute's (NCI) Genomic Data Commons (GDC), a data sharing platform that promotes precision medicine in oncology, provides a rich resource of molecular and clinical data of  almost 13,000 tumor patient samples across 38 different cancer types and subtypes. The platform includes data from The Cancer Genome Atlas (TCGA) and Therapeutically Applicable Research to Generate Effective Treatments (TARGET). The data, which is publicly available, have been utilized by researchers to make novel discoveries and/or validate important findings. To enhance these findings, several important bioinformatics tools to harness genomics cancer data were developed, many of them belonging to the Bioconductor project \cite{gentleman2004bioconductor}. Among those tools is our tool TCGAbiolinks \cite{TCGAbiolinks}, which was developed to facilitate the analysis of TCGA data by incorporating the query, download and processing steps within the Bioconductor project \cite{gentleman2004bioconductor}. This tool allows users to integrate TCGA data with Bioconductor packages thus harnessing a wealth of statistical methodologies for biologically derived data. In addition, it provides integrative methodologies to perform several important downstream analyses, such as DNA methylation and Gene expression integration. A full detailed comparison between TCGAbiolinks and other bioinformatics tools to analyze TCGA data was previously detailed in our report in which we highlight key advantages of using TCGAbiolinks  \cite{TCGAbiolinks}. Although TCGAbiolinks is a suitable R package for most data analysts with a strong knowledge and familiarity with R specifically those who can comfortably write strings of common R commands, we developed TCGAbiolinksGUI to enable user access to the methodologies offered in TCGAbiolinks and to give users the flexibility of point-and-click style analysis without the need to enter specific arguments. TCGAbiolinksGUI takes in all the important features of TCGAbiolinks and offers a graphics user interface (GUI) thereby eliminating any need to familiarize TCGAbiolinks' key functions and arguments.  Tutorials via online documents and YouTube video instructions will assist end-users in taking full advantage of TCGAbiolinks.
Here we present TCGAbiolinksGUI an R/Bioconductor package which uses the R web application framework shiny \cite{shiny} to provide a GUI to process, query, download, and perform integrative analyses of TCGA data.

\subsection{Infrastructure}
The TCGAbiolinksGUI user interface was created using Shiny, a Web Application Framework for R, and uses several packages to provide advanced features that can enhance Shiny apps, such as shinyjs to add JavaScript actions \cite{shinyjs}, shinydashboard to add dashboards \cite{shinydashboard} and shinyFiles \cite{shinyFiles} to provide access to the server file system. 

The following R/Bioconductor packages are used as back-ends for the data retrieval and analysis: TCGAbiolinks \cite{TCGAbiolinks} which allows to search, download and prepare data from the NCI's Genomic Data Commons (GDC) data portal into an R object and perform several downstream analysis;  ELMER (Enhancer Linking by Methylation/Expression Relationship) \cite{yao2015inferring, ELMER2} which identifies DNA methylation changes in distal regulatory regions and correlate these signatures with the expression of nearby genes to identify transcriptional targets associated with cancer; ComplexHeatmap \cite{Gu20052016} to visualize data as oncoprint and heatmaps, pathview \cite{luo2013pathview} which offers pathway based data integration and visualization; and maftools \cite{Maftools} to analyze, visualize and summarize \sigla{MAF}{Mutation Annotation Format} files.


\subsection{Graphical user interface design}
The user interface has been divided into three main \sigla{GUI}{Graphical User Interface} menus. The first menu defines the acquisition of GDC data. The second defines the analysis steps which subdivides according to the molecular data types.  And the third is dedicated to harnessing integrative analyses. We present below a brief description of each menu and their features that can be accessed through a side panel (see figure \ref{fig:fig1}): 


\begin{itemize}
	\item \textbf{GDC Data:} Provides a guided approach to search for published molecular subtype information, clinical and molecular data. In addition, it downloads and processes the molecular data into an R object that can be used for further analysis.
	\item \textbf{Clinical analysis:} Performs survival analysis to quantify and test survival differences between two or more groups of patients and draws survival curves with the 'number at risk' table, the cumulative number of events table and the cumulative number of censored subjects table using the R/CRAN package survminer \cite{survminer}.
	\item \textbf{Epigenetic analysis:} Performs a Differentially methylated regions (DMR) analysis, visualizes the results through both volcano and heatmap plots, and visualizes the mean DNA methylation level by groups.
	\item \textbf{Transcriptomic analysis:} Performs a Differential Expression Analysis (DEA), and visualizes the results through both volcano and heatmap plots. For the genes found as upregulated or downregulated an enrichment analysis can be performed and pathway data can be integrated \cite{luo2013pathview}.
 	\item \textbf{Genomic analysis:} Visualize and summarize the mutations from MAF (Mutation Annotation Format) files through summary plots and oncoplots using the R/Bioconductor maftools package \cite{Gu20052016,Maftools}. % Cite complex heatmap
	\item \textbf{Integrative analysis:} Integrate the DMR and DEA results through a starburst plot. Also, using the DNA methylation data and the gene expression data the R/Bioconductor ELMER package can be used to discover functionally relevant genomic regions associated with cancer \cite{yao2015inferring, ELMER2}.
\end{itemize}

\subsection{Documentation}

We provide a guided tutorial for users via a vignette document which details each step and menu function available at \href{http://bit.do/TCGAbiolinksDocs}{http://bit.do/TCGAbiolinksDocs}, via online documents available at 
\href{http://bit.ly/TCGAbiolinks\_PDFTutorials}{http://bit.ly/TCGAbiolinks\_PDFTutorials}, and via YouTube video instructions showing step by step how each menu works available at \href{http://bit.ly/TCGAbiolinksGUI\_videoTutorials}{http://bit.ly/TCGAbiolinksGUI\_videoTutorials}, which assist end-users in taking full advantage of TCGAbiolinksGUI. A demonstration version of the tool is available at \href{http://tcgabiolinks.fmrp.usp.br:3838/}{http://tcgabiolinks.fmrp.usp.br:3838/}. Users are encouraged to report and file bug reports or feature requests via our GitHub repository \href{https://github.com/BioinformaticsFMRP/TCGAbiolinksGUI/issues}{BioinformaticsFMRP/TCGAbiolinksGUI/issues}. 

\subsection{Docker container}
To further simplify the usability and accessibility of our tool, we provide a docker image compatible with most popular operating system available at \\
\href{https://hub.docker.com/r/tiagochst/tcgabiolinksgui/}{https://hub.docker.com/r/tiagochst/tcgabiolinksgui/}. This file allows users to run TCGAbiolinksGUI without the need to install associated dependencies or configure system files, common steps required to run R installations and load R/Bioconductor packages. 



\section{Results and Discussion}
To provide the users with insights into the usability of our TCGAbiolinksGUI, 1) we compare with other bioinformatics tools currently published in the field; 2) we provide a use-case that allows users a step-by-step guide to analyzing their own cancer molecular data.

\subsection{Comparison of alternative software}

Web tools used for cancer data analysis might be classified into two broad groups. 
The first group only provides an interface to existing software analysis tools.
The Galaxy project (\href{https://galaxyproject.org/}{https://galaxyproject.org/}), which is an open, web-based platform for accessible, reproducible, and transparent computational biomedical research, is an example of such a tool that belongs to this group.
The other group is composed of exploratory tools mainly focused on the visualization of processed data and pre-computed results. The cBioPortal project \cite{gao2013integrative,cerami2012cbio}, by providing several visualizations for mining the TCGA data, is an example of a tool that falls within this classification.

 If one were to classify TCGAbiolinksGUI, it would belong to the first group. Compared to the Galaxy project, TCGAbiolinksGUI offers an open platform which improves the accessibility of R/Bioconductor packages, allowing users an advantage to integrate their features with existing Bioconductor packages without the need to go beyond the R/Shiny frameworks as a common feature from the Galaxy project, which requires the interface elements to be structured through XML files \cite{10.12688/f1000research.9821.1}. 
In addition, going beyond the R/Bioconductor environment requires more software dependencies which make the process to install Galaxy to use R/Bioconductor packages laborious.
On the other hand, compared to cBioPortal, TCGAbiolinksGUI allows users to perform deep integrative analysis by comparing different subtypes of data (i.e. performing an integrative analysis to compare breast cancer samples with a mutation on FOXA1 gene compared to wild-type samples using DNA methylation, gene expression, and motif enrichment analysis on genomic regions of interest). Although cBioPortal offers these features, it would require users to process each step independently and download outside of cBioPortal in order to perform such integrative analysis.


\section{Enhancer Linking by Methylation/Expression Relationships (ELMER)}

Motivated by our discovery of transcriptional enhancers in tissue DNA methylation data \cite{berman2012ng}, and subsequent approaches to linking these enhancers to transcriptional targets using a chromQTL approach \cite{aran2013dna} (reviewed in \citeonline{yao2015review}), we developed the the R/Bioconductor  \textit{ELMER} (Enhancer Linking by Methylation/Expression Relationships) package, a tool which infers regulatory element landscapes and transcription factor networks from cancer methylomes \cite{yao2015inferring}. 

This tool combined DNA methylation and gene expression data from human tissues to infer multi-level cis-regulatory networks through several steps which included the identification of distal enhancer probes with significantly altered DNA methylation levels in primary tumor tissues compared to normal tissues, followed by the identification of putative target genes, and a comprehensive gene regulatory network analysis which combined transcription factor motifs at the altered enhancers with TF expression to identify the underlying master regulators. This approach identified several known and unknown master regulators in TCGA data, such as \sigla{GATA3}{GATA-binding protein 3} and \sigla{FOXA1}{Forkhead box protein A1} in breast cancer, and \sigla{P63}{tumor protein p63} and \sigla{SOX2}{Sex determining region Y-box 2} in squamous cell lung carcinoma \cite{yao2015inferring,silva2016tcga}.

% Present \textit{ELMER}
Based on user feedback and a full review of the source code, we identified and implemented a number of software improvements, which are summarized in table \ref{tab:summary}: (i) The original package contained no standard data structure to handle multiple assays (DNA methylation, gene expression, and clinical data), which would be required for an integrative genomic data analysis. Recently, the Bioconductor team provided such a data structure through the \href{http://bioconductor.org/packages/MultiAssayExperiment/}{MultiAssayExperiment} package. (ii) All auxiliary databases (human TF list, classification of TF in families, gene annotation, DNA methylation annotation and motif occurrences within probe sites) used in the package were created and maintained manually, thereby making the upgrade process laborious; thus, we automated this process. (iii) The package was developed to analyze primary tumor tissue samples compared to normal tissues samples, thus not allowing arbitrary subgroups to be compared (for instance mutants vs. non-mutants, treated vs. untreated, etc.) (iv) Our original approach used known epigenomic markers for enhancers to constrain the genomic regions searched for differential methylation. However, this selection could limit our algorithm to identifying regulatory networks for tissue types that exist in the epigenomic databases; we found this constraint problematic, and thus now search \textit{all} distal regulatory regions without any such filter. (v) The function used to download data from The Cancer Genome Atlas (TCGA) data portal \cite{tomczak2015cancer} broke when the TCGA site was shutdown and its data transferred to The NCI's Genomic Data Commons (GDC) \cite{grossman2016toward}; we now have a more general data provider interface that supports GDC as the default provider. (vi) The package only supported data aligned to Genome Reference Consortium GRCh37 (hg19), and we now provide support for Genome Reference Consortium GRCh38 (hg38). (vii) There was no support to the recent HumanMethylationEPIC (EPIC) array \cite{epic}. In addition to the specific improvements listed above, we substantially re-wrote most of the code to be more efficient and maintainable,  also most of the output plots generated were improved.

\begin{table}[h!]
\centering
\caption{Main differences between ELMER old version (v.1) and the new version (v.2)}
\label{tab:summary}
\begin{tabular}{@{}p{3cm}p{5cm}p{6cm}@{}}
\toprule
\multicolumn{1}{c}{\textbf{Features}} & \multicolumn{1}{c}{\textbf{ELMER Version 1}} & \multicolumn{1}{c}{\textbf{ELMER Version 2}}   \\ \midrule
Primary data structure                   & mee object (custom data structure)                       & MAE object (Bioconductor data structure) \\
Auxiliary data                   & Manually created                       & Programmatically created \\
Number of human TFs                    & 1,982                                  & 1,987 (Uniprot database \cite{apweiler2004uniprot})                 \\
Number of TF motifs                   & 91                                     & 771  (HOCOMOCO v11 database \cite{kulakovskiy2016hocomoco})                 \\
TF classification                     & 78 families                            & 82 families and 331 subfamilies \newline(TFClass database \cite{wingender2013tfclass}) \\
Analysis performed            & Normal vs tumor samples & Group 1 vs group 2                       \\ 
Statistical grouping            & unsupervised only & unsupervised or supervised using labeled groups                       \\ 
TCGA data source                   & The Cancer Genome Atlas (TCGA) (not available)                   & The NCI's Genomic Data Commons (GDC)                                      \\
Genome of reference                   & GRCh37 (hg19)                          & GRCh37 (hg19)/GRCh38 (hg38)          \\
DNA methylation platforms             & HumanMethylation450                                   & HumanMethylationEPIC and HumanMethylation450                                \\
Graphical User interface (GUI)        & None                                   & TCGAbiolinksGUI                       \\
\bottomrule
\end{tabular}
\end{table}

Here, we present a new version of the R \textit{ELMER} package, which addresses all the issues described above. 
The new version of \textit{ELMER} (v2.0.0) is available as an R/Bioconductor package at \burl{https://github.com/tiagochst/ELMER}. And, the new version of \textit{ELMER.data} (v2.0.0), which provides auxiliary data required to perform the analysis, is available at 
\burl{https://github.com/tiagochst/ELMER.data}. 

\subsection*{Implementation}

Here we describe each of following analysis steps shown in figure \ref{fig:elmerworkflow}. For more details, please also check the original ELMER paper \cite{yao2015inferring}. 
\begin{itemize}
    \item Organize data as a \textit{MultiAssayExperiment} object
	\item Identify distal probes with significantly different DNA methylation level when comparing two sample groups.
	\item Identify putative target genes for differentially methylated distal probes, using methylation vs. expression correlation
	\item Identify enriched motifs for each probe belonging to a significant probe-gene pair
	\item Identify master regulatory Transcription Factors (TF) whose expression associate with DNA methylation changes at multiple regulatory regions.
\end{itemize}

Organization of data as a \textit{MultiAssayExperiment} object

To facilitate the analysis of experiments and studies with multiple samples the Bioconductor team created the \href{http://bioconductor.org/packages/SummarizedExperiment/}{\textit{SummarizedExperiment}} class \cite{huber2015orchestrating}, a data structure able to store data and metadata for a single experiment but not for data spanning several experiments for the same sample. To overcome this problem, recently, the MultiAssay SIG (Special Interest Group) created the \href{http://bioconductor.org/packages/MultiAssayExperiment/}{MultiAssayExperiment class} \cite{mae2017} a data structure to manage and preprocess multiple assays for integrated genomic analysis. This data structure is now an input for all main functions of \href{https://github.com/tiagochst/ELMER}{\textit{ELMER}} and can be generated by the \textit{createMAE} function. 



To perform \textit{ELMER} analyses, we need to populate a \textit{MultiAssayExperiment} with a DNA methylation matrix or \textit{SummarizedExperiment} object from HM450K or EPIC platform; a gene expression matrix or SummarizedExperiment object for the same samples; a matrix mapping DNA methylation samples to gene expression samples; and a matrix with sample metadata (i.e. clinical data, molecular subtype, etc.). If TCGA data are used, the last two matrices will be automatically generated.
If using non-TCGA data, the matrix with sample metadata should be provided with at least a column with a patient identifier and another one identifying its group which will be used for analysis, if samples in the methylation and expression matrices are not ordered and with same names, a matrix mapping for each patient identifier their DNA methylation samples and their gene expression samples should be provided to the \textit{createMAE} function.
Based on the genome of reference selected, metadata for the DNA methylation probes, such as genomic coordinates, will be added from   \href{http://zwdzwd.github.io/InfiniumAnnotation}{\citeonline{zhou2016comprehensive}}; 
and metadata for gene expression and annotation is added from Ensembl database \cite{yates2015ensembl} using \href{http://bioconductor.org/packages/biomaRt/}{biomaRt}
\cite{durinck2009mapping}. 

\subsubsection*{Selecting distal probes} 
Probes from HumanMethylationEPIC (EPIC) array and Infinium HumanMethylation450 (HM450) array are removed from the analysis if they have either internal SNPs close to the $3'$ end of the probe; non-unique mapping to the bisulfite-converted genome; or off-target hybridization due to partial overlap with non-unique elements \cite{doi:10.1093/nar/gkw967}. This probe metadata information is
included in \href{https://github.com/tiagochst/ELMER.data}{\textit{ELMER.data}} package, populated from the source file at \url{http://zwdzwd.github.io/InfiniumAnnotation} \cite{doi:10.1093/nar/gkw967}.
To limit ELMER to the analysis of distal elements, probes located in regions of $\pm2 kb$ around transcription start sites (TSSs) were removed.

\subsubsection*{Identification of differentially methylated CpGs (DMCs)}

For each distal probe, samples of each group (group 1 and group 2) are ranked by their DNA methylation beta values, those samples in the lower quintile (20\% samples with the lowest methylation levels) of each group are used to identify if the probe is hypomethylated in group 1 compared to group 2, using an unpaired one-tailed t-test. The 20\% is a parameter to the \textit{diff.meth} function called \textit{minSubgroupFrac}. For the (ungrouped) cancer case, this is set to 20\% as in \citeonline{yao2015inferring}, because we typically wanted to be able to detect a specific molecular subtype among the tumor samples; these subtypes often make up only a minority of samples, and 20\% was chosen as a lower bound for the purposes of statistical power (high enough sample numbers to yield t-test p-values that could overcome multiple hypothesis corrections, yet low enough to be able to capture changes in individual molecular subtypes occurring in 20\% or more of the cases.) This number can be set arbitrarily as an input to the \textit{diff.meth} function and should be tuned based on sample sizes in individual studies. In the \textit{Supervised} mode, where the comparison groups are implicit in the sample set and labeled, the \textit{minSubgroupFrac} parameter is set to 100\%.  An example would be a cell culture experiment with 5 replicates of the untreated cell line, and another 5 replicates that include an experimental treatment.

To identify hypomethylated DMCs, a one-tailed t-test is used to rule out the null hypothesis: $\mu_{group1} \geq \mu_{group2}$, where $\mu_{group1}$ is the mean methylation within the lowest group 1 quintile (or another percentile as specified by the \textit{minSubgroupFrac} parameter) and $\mu_{group2}$ is the mean within the lowest group 2 quintile. Raw p-values are adjusted for multiple hypothesis testing using the Benjamini-Hochberg method \cite{benjamini1995controlling}, and probes are selected when they had adjusted p-value less than $0.01$ (which can be configured using the \textit{pvalue} parameter). For additional stringency, probes are only selected if the methylation difference: $\Delta = \mu_{group1} - \mu_{group2}$ was greater than $0.3$. The same method is used to identify hypermethylated DMCs, except we use the \textit{upper} quintile, and the opposite tail in the t-test is chosen.

\subsubsection*{Identification of putative target gene(s)} 

For each differentially methylated distal probe (DMC), the closest 10 upstream 
genes and the closest 10 downstream genes are tested for inverse correlation between 
methylation of the probe and expression of the gene (the number 10 can be changed using the \textit{numFlankingGenes} parameter). To select these genes, 
the probe-gene distance is defined as the distance from the probe to the transcription 
start site specified by the ENSEMBL gene level annotations \cite{yates2015ensembl} accessed via
the R/Bioconductor package \href{http://bioconductor.org/packages/biomaRt/}{biomaRt} \cite{durinck2009mapping,durinck2005biomart}. By choosing a constant number of genes to test for each probe, our goal is to avoid systematic false positives for probes in gene rich regions. This is especially important given the highly non-uniform gene density of mammalian genomes.
Thus, exactly 20 statistical tests were performed for each probe, as follows. 

For each probe-gene pair, the samples (all samples from both groups) are divided into two 
groups: the M group, which consisted of the upper methylation quintile (the 20\%
of samples with the highest methylation at the enhancer probe), and the U group, 
which consists of the lowest methylation quintile (the 20\% of samples with the 
lowest methylation.) The 20\% ile cutoff is a configurable parameter \textit{minSubgroupFrac} in the \textit{get.pair} function.
As with its usage in the \textit{diff.meth} function, the default value of 20\% is a balance, allowing for the identification of changes in a 
molecular subtype making up a minority (i.e. 20\%) of cases, while also yielding 
enough statistical power to make strong predictions. For larger sample sizes or other experimental designs, this could be set even lower.

For each candidate probe-gene pair, 
the Mann-Whitney U test is used to test the null hypothesis that overall gene 
expression in group M is greater than or equal than that in group U. 
This non-parametric test was used in order to minimize the effects 
of expression outliers, which can  occur across a very wide dynamic range. 
For each probe-gene pair tested, the raw p-value $P_r$ is corrected for multiple 
hypothesis using a permutation approach as follows.
The gene in the pair is held constant, and \textit{x} random methylation probes are 
chosen to perform the same one-tailed U test, generating a set of \textit{x} permutation
p-values $P_p$. We chose the x random probes only from among those that were 
"distal" (farther than $2kb$ from an annotated transcription start site), in order 
to draw these null-model probes from the same set as the probe being tested \cite{sham2014statistical}. 
An empirical p-value $P_e$ value was calculated using the following formula 
(which introduces a pseudo-count of 1):

$$P_e = \frac{num(P_p \leq P_r)+ 1}{x+1}$$

Notice that in the \textit{Supervised} mode, no additional filtering is necessary to ensure that the M and U group segregate by sample group labels.  The two sample groups are segregated by definition, since these probes were selected for their differential methylation, with the same directionality, between the two groups. 



\subsubsection*{Characterization of chromatin state context of enriched probes using FunciVar}

Unlike version 1 of \textit{ELMER}, we now consider \textit{all} distal probes in the identification of regulatory elements. DNA methylation is known to affect several different classes of distal chromatin state element, including active enhancers, poised enhancers, and insulators. In order to provide a functional interpretation of the regulatory elements identified by \textit{ELMER}, we perform a chromatin state enrichment analysis of the probes within significant probe-gene pairs, using the \textit{statePaintR} tools from the \burl{www.statehub.org} \cite{statepaintr}, along with our new FunciVar package \cite{funcivar}. Enrichment of the putative pairs within chromatin states is calculated against a background model that uses the distal probe set that the putative pairs are drawn from. 

\subsubsection*{Motif enrichment analysis}
In order to identify enriched motifs and potential upstream regulatory TFs, all probes with occurring in significant probe-gene pairs are combined for motif enrichment analysis. \sigla{HOMER}{Hypergeometric Optimization of Motif EnRichment} \cite{heinz2010simple} is used to find motif occurrences in a $\pm 250bp$ region around each probe, using HOCOMOCO (HOmo sapiens COmprehensive MOdel COllection) v11 \cite{kulakovskiy2016hocomoco}. Transcription factor (TF) binding models are available at \burl{http://hocomoco.autosome.ru/downloads} (using the HOMER specific format with threshold score levels corresponding to p-value $ \leq 1^{-4}$). 

For each probe set tested (i.e. the set of all probes occurring in significant probe-gene pairs), a motif enrichment Odds Ratio and a 95\% confidence interval are calculated using following formulas:
$$p = \frac{a}{a + b}$$
$$P = \frac{c}{c + d}$$
$$Odds Ration = \frac{\frac{p}{(1-p)}}{\frac{P}{1-P}}= \frac{p(1-P)}{P(1-p)}=\frac{ad}{bc}$$
$$SD = \sqrt{\frac{1}{a} + \frac{1}{b} + \frac{1}{c} + \frac{1}{d}}$$

where $a$ is the number of probes within the selected probe set that contains one 
or more motif occurrences; $b$ is the number of probes within the selected probe 
set that do not contain a motif occurrence; $c$ and $d$ are the same counts within 
the entire array probe set (drawn from the same set of distal-only probes using the same definition as the primary analysis). A probe set was considered significantly enriched 
for a particular motif if the 95\% confidence interval of the Odds Ratio was 
greater than $1.1$ (specified by option \textit{lower.OR}, $1.1$ is default), and the motif 
occurred at least 10 times (specified by option \textit{min.incidence}, $10$ is default) in 
the probe set. 

%\section*{Results} % Optional - only if novel data or analyses are included
%This section is only required if the paper includes novel data or analyses, and should be written as a traditional results section.

\subsubsection*{Identification of master regulator TFs}

When a group of enhancers is coordinately altered in a specific sample subset, this is often the result of an altered upstream \textit{master regulator} transcription factor in the gene regulatory network. \textit{ELMER} tries to identify such transcription factors corresponding to each of the TF binding motifs enriched from the previous analysis step.
For each enriched motif, \textit{ELMER} takes the average DNA methylation of all distal probes (in significant probe-gene pairs) that contain that motif occurrence (within a $\pm 250bp$ region) and compares this average DNA methylation to the expression of each gene annotated as a human TF.

A statistical test is performed for each motif-TF pair, as follows. All samples 
are divided into two groups: the M group, which consists 
of the 20\% of samples with the highest average methylation at all motif-adjacent
probes, and the U group, which consisted of the 20\%  of samples with the lowest 
methylation. This step is performed by the \textit{get.TFs} function, which takes \textit{minSubgroupFrac} as an input parameter, again with a default of 20\%.
For each candidate motif-TF pair, the Mann-Whitney U test is used to test 
the null hypothesis that overall gene expression in group M is greater or equal 
than that in group U. This non-parametric test was used in order to minimize the 
effects of expression outliers, which can occur across a very wide dynamic range. 
For each motif tested, this results in a raw p-value ($P_r$) for each of the human TFs.
All TFs are ranked by their $-log_{10}(Pr)$ values, and those falling within the top 5\% of 
this ranking were considered candidate upstream regulators. The best upstream 
TFs which are known to recognize to specific binding motif are automatically extracted as putative 
regulatory TFs, and rank ordered plots are created to visually inspect these relationships, as shown in the example below. Because the same motif can be recognized by many transcription factors of the same binding domain family, we define these relationships at both the family and subfamily classification level using the 
classifications from TFClass database \cite{wingender2013tfclass}. Use of this database is a major change from version 1 of ELMER, which used custom curations for DNA binding domain families. Use of the TFClass database is preferable because it is well curated and regularly updated to reflect new findings.



%\begin{landscape}
\tikzstyle{container} = [
    rectangle,
    draw,
    inner sep=0.2 cm,
    dashed
]
\tikzstyle{start} = [circle,
					 minimum size=2mm,
                     rounded corners=3mm,
					 very thick,
                     draw=green!50!black,
                     top color=green!50!black,
                     bottom color=green!50!black, 
                     text=white,
                     font=\tiny]

\tikzstyle{end} = [circle,
				  minimum size=2mm,
                  rounded corners=3mm,
                  very thick,draw=red!50!black, 
                  top color=red!50!black,
                  bottom color=red!50!black, 
                  text=white,
                  font=\tiny]

\tikzstyle{function} = [rectangle,
						minimum size=6mm,
                        rounded corners=3mm,
                        very thick,
                        draw=black!50, 
                        top color=white,
                        bottom color=white,
                        font=\itshape\footnotesize]

\tikzstyle{datain} = [
	rectangle, 
	rounded corners, 
    minimum width=3cm, 
    minimum height=0.5cm,
    text centered,
    font=\footnotesize, 
    draw=green!50!black, 
    fill=white, 
    text=black
]
                      
\tikzstyle{dataaux} = [
	rectangle, 
    rounded corners, 
    minimum width=3cm, 
    minimum height=0.5cm,
    text centered,
    font=\footnotesize,
    draw=orange, 
    fill=white, 
    text=black
]
                       
\tikzstyle{dataout} = [
	rectangle, 
	rounded corners, 
    minimum width=3cm, 
    minimum height=0.5cm,
    text centered,
    font=\footnotesize, 
    draw=blue, 
    fill=white, 
    text=black
]

% Pacakge labels
\tikzstyle{arrow} = [
	thick,
    ->,
    >=stealth,
    -latex',
    draw,
    rounded corners
]

\tikzstyle{labelelmer}=[
	rectangle,
    draw,
    fill=black!50!red,
    draw = black,
    minimum width=450pt,
    minimum height=1.5em,
    text=white,
    rotate = 90, 
    label={[rotate=90]center:\textcolor{white}{\textbf{ELMER package}}}
]

\tikzstyle{labeltcgabiolinks}=[
	rectangle,
	draw,
    fill=black!50!blue,
    draw = black,
    minimum width=420pt,
    minimum height=1.5em,
    text = green,
    rotate = 90, 
    label={[rotate=270]center:\textcolor{white}{\textbf{TCGAbiolinks/TCGAbiolinksGUI packages}}}
]


\tikzstyle{labelfuncivar}=[
	rectangle,
	draw,
    fill=black!20!orange,
    draw = black,
    xshift = -0.0cm,
    minimum width=480pt,
    minimum height=1.5em,
    text=white,
    rotate = 0, 
    label={[rotate=0]center:\textcolor{white}{\textbf{StateHub/StatePaintR/funcivar package}}}
]
\tikzstyle{labelgdc}=[
	rectangle,
	draw,
    fill=black!50!gray,
    draw = black,
    minimum width=167pt,
    minimum height=1.5em,
    text=white,
    yshift = 0.10cm,
    xshift = 0.1cm,
    rotate = 0, 
    label={[rotate=0]center:\textcolor{white}{\textbf{GDC database}}}
]
\tikzstyle{every annotation}=[fill=white, font=\sf \small, scale=0.5, text width=4cm, inner sep=2mm, text=black,draw = orange]


\begin{figure}[!ht]
\centering
  \resizebox{0.95\textwidth}{!}{%
\begin{tikzpicture}[node distance = 1.5cm, auto, shorten >=1pt,thick,font=\itshape\footnotesize]
\linespread{0.8}{
%\node (start) [start] {START};
\node (func1) [function, yshift = -0.5cm] {\textit{createMAE}};
\node [datain, right of=func1, yshift = 0.5cm, xshift = 2cm] (dna) {DNA methylation object};
\node [datain, right of=func1, yshift = -0.5cm, xshift = 2cm] (exp) {Gene expression object};
\node (out1) [dataout, below of=func1, yshift = -0.3cm,text width=3cm] {Multi Assay Experiment object};
\node (func2) [function, below of = out1] {get.diff.meth};
%\node (out2) [dataout, below of=func2, yshift = 0.3cm] {List of differently methylated probes};
\node (func3) [function, below of=func2] {GetNearGenes};
%\node (out3) [dataout, below of=func2, yshift = 0.3cm] {List of near genes for differently methylated probes};
\node (func4) [function, below of=func3] {get.pair};
%\node (out4) [dataout, below of=func4, yshift = 0.3cm] {List of pairs: differently expressed gene and differently methylated probes};
\node (func5) [function, below of=func4, yshift = -0.5cm] {get.enriched.motif};
%\node (out5) [dataout, below of=func5, yshift = 0.3cm] {List of enriched motifs};
\node (func6) [function, below of=func5] {get.TFs};
\node (func7) [function, below of=func6,yshift = -0.5cm] {TF.survival};
%\node (out5) [dataout, below of=func5, yshift = 0.3cm] {List of regulator};
\node [dataaux, left of=func5, xshift =-3cm] (elmerdata1) {Probes.motif};
%\node [dataaux, above of=elmerdata1] (enhancer) {enhancer};
\node [dataaux, left of=func6, yshift = 0.0cm, xshift =-3cm] (elmerdata2) {motif.relevant.TFs};
\node [dataaux, left of=func6, yshift = -1.0cm, xshift =-3cm] (elmerdata3) {human.TFs};
%\node (end) [end, below of=func7] {END};
\node (func8) [function, left of=func1,yshift = -3cm,xshift = -3cm] {get.feature.probe};
\node (probes) [datain, left of=func1,xshift = -3cm] {distal probes};
\node [dataaux, below of=func8] (tss) {ENSEMBL TSS};
\node [dataaux, below of=tss] (probesmetadata) {Probes metadata};


% funcvat
\node (funciVar) [function, below of=func7, xshift = 3cm, yshift = -1.4cm] {enrich.segments};
\node [dataaux, left of=funciVar,xshift = -2cm] (statehub) {Statehub tracks};
%\node [dataaux, left of=statehub,xshift = -2cm, yshift = 0.2cm] (encode) {ENCODE};
%\node [dataaux, left of=statehub,xshift = -2cm, yshift = -0.4cm] (roadmap) {ROADMAP};
%\node [dataaux, left of=statehub,xshift = -2cm, yshift = 0.8cm] (blueprint) {BLUEPRINT};
%\draw [arrow,dashed,draw=orange] (encode.east) -- (statehub.west);
%\draw [arrow,dashed,draw=orange] (roadmap.east) -- (statehub.west);
%\draw [arrow,dashed,draw=orange] (blueprint.east) -- (statehub.west);

\draw [arrow,dashed,draw=orange] (statehub.east) -- (funciVar.west);

\draw [arrow] (func4) -- ++(4.9,0) -- ++(0,-1) |- node {} (funciVar);

% Draw edges
%\path [arrow] (start) -- (func1);
\path [arrow,dashed,draw=green!50!black] (dna) |- (func1);
\path [arrow,dashed,draw=green!50!black] (exp) |- (func1);
\draw [arrow,dashed,draw=blue] (out1.west) -- ++(-.5,0) -- ++(0,-1) |- (func4.west);
\draw [arrow,dashed,draw=blue] (out1.west) -- ++(-.5,0) -- ++(0,-1) |- (func6.west);
\draw [arrow] (func1) -- (out1);
\draw [arrow] (out1) -- (func2);
\draw [arrow] (func2) -- node {} (func3);
\draw [arrow] (func3) -- (func4);
\draw [arrow] (func4) -- node {} (func5);
\draw [arrow] (func5) -- node {}(func6);
\draw [arrow] (func6) -- (func7);
%\draw [arrow] (func7) -- (end);
\draw [arrow,dashed,draw=orange] (elmerdata1) -- node {} (func5);
\draw [arrow,dashed,draw=orange] (elmerdata2.east) -- (func6.west);
\draw [arrow,dashed,draw=orange] (elmerdata3.east) -- ++(.5,0) -- ++(0,0.2) |- (func6.west);
%\draw [arrow,dashed,draw=orange] (enhancer.north) -- (func8.south);
\draw [arrow,dashed,draw=orange] (tss.north) -- (func8.south);
\draw [arrow,dashed,draw=orange] (probesmetadata.east) -- ++(.1,0) -- ++(0,0.2) |- (func8.east);
\path [arrow,dashed,draw=green!50!black] (probes) -- (func1);
\draw [arrow] (func8.north)  --  (probes);

% Containers
\node [container, 
       fit=(exp)(dna)(func1)(probes), 
       label={[font=\scriptsize,anchor=east] west:Data input}]
       (container1){};
\node [container, 
	   fit=(func2), 
       label={[font=\scriptsize,anchor=west,name=lfunc1] east:{\parbox[c]{4.0cm}{Identifying differentially\\ methylated probes}}}]
       (container2){};
\node [container, 
       fit=(func3)(func4), 
	   label={[font=\scriptsize,anchor=west,name=lfunc2] east:{\parbox[c]{4.0cm}{Identifying putative \\probe-gene pairs}}}]
       (container3){};
\node [container, 
 	   fit=(func5), 
       label={[font=\scriptsize,anchor=west] east:{\parbox[c]{4.0cm}{Motif enrichment\\ analysis}}}]
       (container4){};
\node [container, 
       fit=(func6), 
       label={[font=\scriptsize,anchor=west] south east:Identifying regulatory TFs}]
       (container5){};
\node [container, 
       fit=(elmerdata1)(elmerdata1), 
       label={[name=l1,font=\scriptsize,anchor=east] west:ELMER.data}]
       (container6){};
%\node[draw,text width=3cm, above of = elmerdata1]{ELMER.data};
\node [container, 
	   fit=(func8)(probesmetadata), 
	   label={[name=l3,font=\scriptsize,anchor=east] west:{\parbox[r]{2.0cm}{Select probes \\$\pm 2Kb$  distant \\ from TSS}}}]
       (container8){};

\node [container, 
       fit=(elmerdata2), 
       label={[name=l2,font=\scriptsize,anchor=east] west:TFClass database}]
       (container7){};
\node [container, 
	   fit=(elmerdata3), 
	   label={[name=l3,font=\scriptsize,anchor=east] west:Uniprot database}]
       (container8){};
\node [draw,  
       minimum height=450pt,
	   minimum width=450pt,
       fit=(l1)(exp)(dna)(elmerdata3)(l2)(lfunc1)(lfunc2)]
       (container9){};
\node at (container9.west) [labelelmer] {};
  
%------------------------------ TCGAbiolinks
\node (GDCprepare) [function, right of = func1, yshift =-1.3cm,xshift =8.8cm] {\textit{GDCprepare}};
\node (GDCdownload) [function, above of = GDCprepare,yshift =-0.4cm] {\textit{GDCdownload}};
\node (GDCquery) [function, above of = GDCdownload,yshift =-0.4cm] {\textit{GDCquery}};
\node (TCGAanalysesurvival) [function, right of = func7,xshift =7.8cm] {\textit{TCGAanalyse\_survival}};
\node (TCGAanalyzeEAcomplete) [function, right of = func4,yshift =0.4cm,xshift =7.8cm] {\textit{TCGAanalyze\_EAcomplete}};
\node (TCGAanalyzePathview) [function, right of = func4,yshift =-0.7cm,xshift =7.8cm] {\textit{TCGAanalyze\_Pathview}};
\node (TCGAvisualizeoncoprint) [function, right of = func4,yshift =-1.8cm,xshift =7.8cm] {\textit{TCGAvisualize\_oncoprint}};

\draw [arrow] (GDCquery) -- node {}(GDCdownload);
\draw [arrow] (GDCdownload) -- (GDCprepare);
\draw [arrow] (GDCprepare.west) -- ++(-0.3,0) -- ++(0,0.2) |- (dna.east);
\draw [arrow] (GDCprepare.west) -- ++(-0.3,0) -- ++(0,0.2) |- (exp.east);

\node (subtypeinfo) [dataaux, below of = GDCprepare,yshift =0.6cm] {Subtype information};
\node (molecularinfo) [dataaux, below of = subtypeinfo,yshift =0.6cm] {Molecular data};
\node (clinicalinfo) [dataaux,  below of = molecularinfo,yshift =0.6cm] {Clinical data};
\node (mafinfo) [dataaux, below of = clinicalinfo,yshift =0.6cm] {Mutation data};

\draw [arrow,dashed,draw=orange] (mafinfo.east) -- ++(0.5,0) -- ++(0,-0.2) |-    (TCGAvisualizeoncoprint.east);
\draw [arrow,dashed,draw=orange] (clinicalinfo.east)  -- ++(0.3,0) -- ++(0,0.2) |-   (GDCprepare.east);
\draw [arrow,dashed,draw=orange] (subtypeinfo.east)   -- ++(0.2,0) -- ++(0,0.2) |-   (GDCprepare.east);
\draw [arrow,dashed,draw=orange] (molecularinfo.east) -- ++(0.3,0) -- ++(0,0.2) |-  (GDCprepare.east);
\node [draw,  
       minimum height=420pt,
       minimum width=170pt, 
       xshift = 0.25cm,
       yshift = -0.25cm,
       fit=(TCGAanalyzeEAcomplete)(GDCquery)(TCGAanalysesurvival)(clinicalinfo)(subtypeinfo)](container10){};
\node at (container10.east) [labeltcgabiolinks] {};
\draw [latex'-latex',double] (TCGAanalysesurvival) --  (func7);
\draw [arrow] (func4.east)  -- ++(4.4,0) -- ++(0,0.2) |-  (TCGAanalyzeEAcomplete);
\draw [arrow] (func4.east)  -- ++(4.4,0) -- ++(0,-0.2) |-  (TCGAanalyzePathview);
\draw [arrow] (func4.east)  -- ++(4.4,0) -- ++(0,-0.2) |-  (TCGAvisualizeoncoprint);
\draw [arrow] (func6.east)  -|   (TCGAvisualizeoncoprint.south);
%------------------------------ 
\node [labelgdc, above of = GDCquery,xshift=-0.30cm,yshift=-0.05cm] (gdc) {};
\draw [latex'-latex',double] (GDCquery) --  (gdc.300);

\draw [draw,dashed] (gdc.188) |- (GDCdownload.west);
\draw [arrow,dashed] (gdc.188) |- (mafinfo.west);
\draw [arrow,dashed] (gdc.188) |- (clinicalinfo.west) ;
\draw [arrow,dashed] (gdc.188) |- (molecularinfo.west) ;
}

\tikzstyle{labelencode}=[
	rectangle,
	draw,
    fill=black!50!gray,
    draw = black,
    minimum width=150pt,
    minimum height=1.5em,
    text=white,
    yshift = 0.10cm,
    xshift = 0.1cm,
    rotate = 0, 
    label={[rotate=0]center:\textcolor{white}{\textbf{ENCODE database}}}
]
\tikzstyle{labelroadmap}=[
	rectangle,
	draw,
    fill=black!50!gray,
    draw = black,
    minimum width=150pt,
    minimum height=1.5em,
    text=white,
    yshift = 0.10cm,
    xshift = 0.1cm,
    rotate = 0, 
    label={[rotate=0]center:\textcolor{white}{\textbf{ROADMAP database}}}
]
\tikzstyle{labelblueprint}=[
	rectangle,
	draw,
    fill=black!50!gray,
    draw = black,
    minimum width=150pt,
    minimum height=1.5em,
    text=white,
    yshift = 0.10cm,
    xshift = 0.1cm,
    rotate = 0, 
    label={[rotate=0]center:\textcolor{white}{\textbf{BLUEPRINT database}}}
]

\node [draw,  
       minimum height=6.52em,
       minimum width=480pt, 
       xshift = 1.60cm,
       yshift = 0.05cm,
       fit=(funciVar)(funciVar)](containerFunciVar){};
\node at (containerFunciVar.south) [labelfuncivar] {};


\node [labelencode, right of = statehub,xshift=-8.20cm,yshift=-0.15cm] (encode) {};
\node [labelroadmap, below of = encode,xshift=-0.1cm,yshift=0.7cm] (roadmap) {};
\node [labelblueprint, above of = encode,yshift=-0.8cm,xshift=-0.1cm] (blueprint) {};
%\draw [latex'-latex',double] (encode.180) --  (containerFunciVar.0);
\draw [double,->] (encode.0) --  (statehub.180);
\draw [double,->] (roadmap.0) -- ++(1.4,0) |-   (statehub.180);
\draw [double,->] (blueprint.0) -- ++(1.4,0) |-  (statehub.180);
\end{tikzpicture}
  }%
  
  \caption{ELMER workflow: ELMER receives as input a DNA methylation object, a gene expression object (a matrix or a SummarizedExperiment object) and a Genomic Ranges (GRanges) object with distal probes to be used as filter which can be retrieved using the \textit{get.feature.probe} function. The function \textit{createMAE}  will create a Multi Assay Experiment object keeping only samples that have both DNA methylation and gene expression data. Genes will be mapped to genomic position and annotated using ENSEMBL database \cite{doi:10.1093/database/baw093}, while for probes it will add annotation from \citeauthor{doi:10.1093/nar/gkw967} (\href{http://zwdzwd.github.io/InfiniumAnnotation}{http://zwdzwd.github.io/InfiniumAnnotation}) . This MAE object will be used as input to the next analysis functions. First, it identifies differentially methylated probes followed by the identification of their nearest genes (10 upstream and 10 downstream) through the  \textit{get.diff.meth} and  \textit{GetNearGenes} functions respectively. For each probe, it will verify if any of the nearby genes were affected by its change in the DNA methylation level and a list of  gene and probes pairs will be outputted from \textit{get.pair} function. For the probes in those pairs, it will search for enriched regulatory Transcription Factors motifs with the  \textit{get.enriched.motif} function. Finally, the  enriched motifs will be correlated with the level of the transcription factor through the \textit{get.TFs} function. In the figure green Boxes represents user input data, blue boxes represents output object, orange boxes represent auxiliary pre-computed data and gray boxes are functions.}
  \label{fig:elmerworkflow}
\end{figure}
%\end{landscape}
