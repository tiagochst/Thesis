\section{Conclusions}


The main goal of this project has been to develop tools and workflows to perform integrative analysis as well as their application in the analysis of public cancer data. The main motivation was that the integration of data from different databases, which are stored in different formats and with different biological meanings, is both a complex process from the computational point of view, because dealing with a large amount of data requires the optimization of the use of computational resources, as well as from the biological point of view, because it is a biological process still under study, often unknown, or even without a scientific consensus.

\section{Conclusions and future studies}

In this section, we list the conclusions and future studies for the results reported in the previous chapters.

\subsection{Conclusions and future works of TCGAbiolinks}

This work was an effort towards the search, download, and
processing of data from the NCI's Genomic Data Commons (GDC) data portal for
downstream analysis. It offered several single-dataset
exploratory analysis such as clustering methods, box plots,
 differential gene expression analysis, differential methylated CpGs (DMCs) analysis,
 survival analysis, and some integrative analysis
  such as the correlation of the regions differently methylated with levels of expression of the nearest gene to identify those affected by the DNA methylation, and gene set enrichment analysis.
As future works, the improvement of the integrative analysis performed by the starburst plot is suggested. This method which integrates the results from two different analysis DEA (differential expression analysis) and DMR (differential methylated regions), should compare the gene expression and DNA methylation of the same sample instead of using the average levels of all samples in a group.

\subsection{Conclusions and future works of TCGAbiolinksGUI}

This work created a Graphical User Interface (GUI) to our command-line tools to help users without programming knowledge to perform deeper downstream analysis.Possible improvements in future work include improving the import of external user data and improving the tool modularization, so that the addition of new menus and packages are facilitated.

\subsection{Conclusions and future works of ELMER}

This work was an effort towards an integrative analysis
using RNA-seq, DNA methylation, and histone marks to identify a candidate regulatory network.
ELMER identifies distal regions with a difference in DNA methylation and correlates them
with the gene expression levels of upstream and downstream genes.
A motif enrichment analysis is performed on the regulatory regions that affect the expression of distant genes to identify potential
regulatory TF candidates. As future work, we suggest expanding the algorithm to consider promoter regions,  accepting \sigla{WGBS}{whole-genome bisulfite sequencing} data in  DNA methylation analysis, and using mutation information instead of DNA methylation
to identify regulatory regions mutated that might have affected the regulation of the upstream/downstream genes.

\section{Publications, presentations, and software of the Doctorate Period}

The work produced during the Doctorate period in form of scientific articles,
software and conference presentations are shown in the next subsections.
The published scientific articles were divided into two groups, one with the first authorship ones, and the other with the co-authorship ones. Those are listed in the following subsection: "First-authored papers", "Co-authored papers","First-authored software" and "Co-authored software".
The subsection "workshops and workflows" contains all the material created to help users to use the created tools. Finally, the subsection "Conferences \& presentations" list all oral and poster presentations at international and national conferences.

\subsection{First-authored papers}
\begin{itemize}
	\item \citetext{TCGAbiolinks}
\begin{itemize}
	\item 	The main contributions to the work were the structuring of the package according to the standards of the Bioconductor project, the creation of all data functions (query, download and prepare) and some  function for analysis and visualization (survival analysis, DNA methylation analysis and plots and integration of DNA methylation and gene expression and visualization in a starburst plot function), the creation of unitary tests, the creation of the documentation for the functions listed above, and the creation of two use case.
\end{itemize}

	\item \citetext{10.12688/f1000research.8923.2}
	\begin{itemize}
  \item The main contributions to the work were the development and testing of the sections “Experimental data”, “DNA methylation analysis”, “Motif analysis” and “Integrative analysis”, the creation and maintenance of the workflow version available in the Bioconductor website.
	\end{itemize}
	\item \citetext{Silva147496}
	\begin{itemize}
  \item The main contributions to the work were
	the creation of the graphical user interface structure and all the menus except the "Transcriptomic analysis - Network of inference and differential expression analysis" menus, the creation of the documentation, docker image, and tutorials (PDF and youtube videos).
	\end{itemize}
	\item \citetext{ELMERv2}
	\begin{itemize}
		\item 	The main contributions to the work were the code  refactoring  to provide greater stability, performance, and ease of use, the alteration of the main data structure to a standard Bioconductor data structure, the integration of the ELMER and TCGAbiolinks tools to import data from GDC, the creation of a graphical user interface and interactive HTML reports, the expansion of the algorithm to consider a supervised analysis mode, the creation of the case of study in the article,  the restructuring of all the documentation by changing from a PDF format to an HTML and the addition of  unit tests to the tool.
	\end{itemize}

\end{itemize}

\subsection{Co-authored papers}

\begin{itemize}
	\item \citetext{Lingutjnl-2017-314607}
	\begin{itemize}
  \item The main contribution to the work was to perform the integrative analysis of esophageal cancer using the ELMER package.
	\end{itemize}
	\item \citetext{cell}
	\begin{itemize}
  \item  The main contributions to the work were the validation of the results found by reanalyzing all DNA methylation data and clinical data. In addition, the analysis methods used in this article have been made available in the TCGAbiolinks package.
	\end{itemize}
	\item \citetext{malta2017glioma}
	\begin{itemize}
	\item  The main contribution to the work was to help in writing about bioinformatics tools and analysis.
	\end{itemize}
  \item \citetext{cava2017spidermir}
	\begin{itemize}
	\item The main contributions to the project were the integration of the tools TCGAbiolinks and SpidermiR, the structuring of the code as a package and help in the creation of the documentation.
	\end{itemize}
  \item \citetext{foxj1}
	\begin{itemize}
	\item The main contributions to the study were differential gene expression analysis for samples that had high expression of the FOXJ1 gene compared to those with low expression and the integration of the results with copy number alteration and mutation data using TCGAbiolinks.
	\end{itemize}
\end{itemize}


\subsection{First-authored software}
\begin{description}
	\item[TCGAbiolinks (version 2.0)] An R/Bioconductor package for integrative analysis of TCGA data. Published in Bioconductor \burl{http://bioconductor.org/packages/TCGAbiolinks/}. Source code available in GitHub \burl{https://github.com/BioinformaticsFMRP/TCGAbiolinks}.
    \item[TCGAbiolinksGUI] A Graphical User Interface to analyze cancer genomics and epigenomics data. Published in GitHub \burl{https://github.com/BioinformaticsFMRP/TCGAbiolinksGUI}.
    \item[ELMER (version 2.0)] Enhancer Linking by Methylation/Expression Relationship (ELMER) is package to identify tumor-specific changes in DNA methylation within distal enhancers, and link these enhancers to downstream target genes. Published in Bioconductor \burl{http://bioconductor.org/packages/ELMER/}. Source code available in GitHub  \burl{https://github.com/tiagochst/ELMER}.
\end{description}

\subsection{Co-authored software}
\begin{description}
	\item[SpidermiR: An R/Bioconductor package for integrative network analysis with miRNA data]{Published in Bioconductor \burl{http://bioconductor.org/packages/SpidermiR/}. Source code available in GitHub \burl{https://github.com/claudiacava/SpidermiR}}.
\end{description}


\subsection{Workshops and workflows}
\begin{description}
\item[Workshop]{Integrative analysis workshop with TCGAbiolinks and ELMER. Dana Farber Cancer Institute, Boston, MA. Link to workshop: \burl{https://bioinformaticsfmrp.github.io/Bioc2017.TCGAbiolinks.ELMER/index.html}. Source code available in GitHub \burl{https://github.com/BioinformaticsFMRP/Bioc2017.TCGAbiolinks.ELMER}.}
\item[Workflow]{TCGA Workflow: Analyze cancer genomics and epigenomics data using Bioconductor packages. Available at \burl{https://www.bioconductor.org/help/workflows/TCGAWorkflow/}. Source code available in GitHub \burl{https://github.com/BioinformaticsFMRP/TCGAWorkflow}.}

\end{description}

\subsection{Conferences \& presentations}
\begin{description}
	\item[SNOLA 2016 - Update on Neuro-Oncology - Oral presentation - 04/19/2016]{"EPIGENOMIC AND TRANSCRIPTOMIC ANALYSIS OF ADULT GLIOMA REVEALS CANDIDATE DRIVER TRANSCRIPTION FACTORS INVOLVED IN GLIOMA PROGRESSION." Windsor Barra Hotel, Rio de Janeiro - Brazil}
    \item[Chromatin and Epigenetics - Poster presentation - 5/5/2017]{TCGAbiolinksGUI: A Graphical User Interface to analyze cancer genomics and epigenomics data. EMBL, Heidelberg, Germany}
	\item[Omics Seminar - Oral presentation - 06/06/2017]{Enhancer Linking by Methylation/Expression Relationship: a case study using Breast Cancer. Cedars-Sinai Medical Center, Los Angeles, California.}
	\item[Bioc2017 - Oral presentation - 07/28/2017]{Workshop: Integrative analysis workshop with TCGAbiolinks and ELMER. Dana Farber Cancer Institute, Boston, MA. Link to workshop: \burl{https://bioinformaticsfmrp.github.io/Bioc2017.TCGAbiolinks.ELMER/index.html}}
\item[From Single to Multiomics - Poster presentation - 11/13/2017]{Enhancer Linking by Methylation/Expression Relationships with the R package (ELMER) version 2. EMBL, Heidelberg, Germany}

\end{description}


%\bibliography{references}
