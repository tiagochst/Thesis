\section{Conclusions}


The main goal of this project has been to develop tools and workflows to perform integrative analysis as well as their application in the analysis of public cancer data. The motivations came from the fact that integrating data from different databases, stored in different formats and with different biological meanings is a complex process from both a computational point of view by dealing with big datasets which requires an optimized use of computational resources as well as from a biological point of view for dealing with a biological process still under study, often unknown, or even without a scientific consensus.

\section{Conclusions and future studies}

In this section, we list the conclusions and future studies for the results reported in the previous chapters.

\subsection{Conclusions and future works of TCGAbiolinks}

This work has presented an effort towards the search, download and
prepare of data from the NCI's Genomic Data Commons (GDC) data portal for
downstream analysis. It offered several single-dataset
exploratory analysis such as PCA, clustering methods, box plots,
 differential gene expression analysis, differential methylated CpGs (DMCs) analysis,
 survival analysis and some integrative analysis
 such as the integration of differential methylated CpGs results with  differential gene expression analysis
 to identify regions different methylated with the nearest gene level expression changed, and
 gene set enrichment analysis.
As future works, it is required the improvement of integrative analysis of the starburst plot, this method
integrates the results from two different analysis DEA (differential expression analysis) and DMR
(differential methylated regions).
There is, however, a problem that the results are analyzed considering the set and not the same sample,
that means the expression and methylation data of a sample are not compared to each other.
It is inferred if a data set has a differentiated average of methylation and a mean of
expression of the nearest differentiated gene, and may even be different populations.
This flaw is corrected in ELMER algorithms.

\subsection{Conclusions and future works of TCGAbiolinksGUI}

This work created a Graphical User Interface (GUI) to our command-line tools,
with the purpose to help users without programming knowledge to perform a
deeper downstream analysis. Among possible improvements in future works are
the export of figures in vector formats (PDF, SVG) that do not lose quality
if altered (enlarged or reduced), the facilitation of the incorporation of external
user data, which although can be incorporated if formatted in the currently defined standard,
this can still be improved, the creation of a modularization of the tool,
which would load only the packages chosen by the user in order to decrease the
 number of libraries needed to run the interface, which
would be a challenge and may not be possible due to the limitations of the development tools used R/Shiny.

\subsection{Conclusions and future works of ELMER}

This work has presented an effort towards the integrative analysis performed
using RNA-seq, DNA methylation, and histone marks to identify a candidate regulatory network.
It mainly identifies distal regions with a difference in DNA methylation and correlates them
with the gene expression levels of upstream and downstream genes.
A motif enrichment analysis is performed in the regulatory regions from the anti-correlated
pairs (loss of DNA methylation and gain o gene expression) to identify potential
regulatory TF candidates. As future works we suggest to expand the algorithm to
the promoter regions, to expand the DNA methylation analysis to accept
\sigla{WGBS}{whole-genome bisulfite sequencing} data
 and to use mutation information instead of DNA methylation
to identify regulatory regions mutated that might have affected the
regulation of a upstream/downstream genes.


\section{Publications, presentations and softwares of the Doctorate Period}

The work produced during the Doctorate period in form of scientific articles,
softwares and conference presentations is shown in the next subsections.
The published scientific articles were divided into two groups, one with the  first authorship ones, and the other with the co-authorship ones. Those are listed in the following subsection: "First-authored papers", "Co-authored papers","First-authored softwares" and "Co-authored softwares".
The subsection "workshops and workflows" contains all the material created to help users to use the developed tools. Finally, the subsection "Conferences \& presentations" list all oral and poster presentations in international and national conferences.

\subsection{First-authored papers}
\begin{itemize}
	\item \citetext{TCGAbiolinks}
\begin{itemize}
	\item 	Main contribuitions to the paper and tool: responsible for structuring of the package according to the standards of the Bioconductor project, creation of all data functions (query, download and prepare), some  function for analysis and visualization (survival analysis, DNA methylation analysis and plots and integration of DNA methylation and gene expression and visualization in a starburst plot function), creation of unitary tests, creation of the documentation and content of all functions listed above, and the creation of half of the  use case presented in the paper and package.
\end{itemize}

	\item \citetext{10.12688/f1000research.8923.2}
	\begin{itemize}
  \item Main contribuitions to the paper and tool: Responsible for the
	development and testing the sections “Experimental data”, “DNA methylation analysis”, “Motif analysis” and “Integrative analysis”, creating and maintaining the workflow version available in the Bioconductor website.
	\end{itemize}
	\item \citetext{Silva147496}
	\begin{itemize}
  \item Main contribuitions to the paper and tool: Responsible for the creation of the graphical user interface structure and all the menus except the "Transcriptomic analysis - Network of inference and differential expression analysis" menus, creation of the documentation, docker image and tutorials (PDF and youtube videos).
	\end{itemize}
	\item \citetext{ELMERv2}
	\begin{itemize}
		\item 	Main contribuitions to the paper and tool: responsible for re-writting the code to provide greater stability, performance, and ease of use, changing the main data strucuture to a standard Bioconductor data structures, integrating ELMER and TCGAbiolinks for data import from GDC, creating graphical user interface, creation of an interactive HTML reports, expansion of the algorithm to consider supervised cases, creation of the case of study in the article, restructuring of all the documentation by changing from a PDF format to an HTML and adding unit tests to the tool.
	\end{itemize}

\end{itemize}

\subsection{Co-authored papers}

\begin{itemize}
	\item \citetext{Lingutjnl-2017-314607}
	\begin{itemize}
  \item Main contribuitions to the paper: performed integrative analysis of esophageal cancer using the ELMER package.
	\end{itemize}
	\item \citetext{cell}
	\begin{itemize}
  \item Main contribuitions to the paper: All the DNA methylation and clinical data used in the paper was redownloaded and re-analysed to validate the findings. Also the methods of this article were made available in the TCGAbiolinks package.
	\end{itemize}
	\item \citetext{malta2017glioma}
	\begin{itemize}
	\item Main contribuitions to the paper: helped to write the text about bioinformatic tools and analysis.
	\end{itemize}
  \item \citetext{cava2017spidermir}
	\begin{itemize}
	\item Main contribuitions to the paper: responsible for integration of TCGAbiolinks tool to the SpidermiR, help with package and documentation strucutre.
	\end{itemize}
  \item \citetext{foxj1}
	\begin{itemize}
	\item Main contribuitions to the paper: performed differential expression analysis for high expressed FOXJ1 samples vs low expressed, integrated results with copy number alteration and mutation data using TCGAbiolinks.
	\end{itemize}
\end{itemize}


\subsection{First-authored softwares}
\begin{description}
	\item[TCGAbiolinks (version 2.0)] An R/Bioconductor package for integrative analysis of TCGA data. Published in Bioconductor \burl{http://bioconductor.org/packages/TCGAbiolinks/}. Source code available in GitHub \burl{https://github.com/BioinformaticsFMRP/TCGAbiolinks}.
    \item[TCGAbiolinksGUI] A Graphical User Interface to analyze cancer genomics and epigenomics data. Published in GitHub \burl{https://github.com/BioinformaticsFMRP/TCGAbiolinksGUI}.
    \item[ELMER (version 2.0)] Enhancer Linking by Methylation/Expression Relationship (ELMER) is package to identify tumor-specific changes in DNA methylation within distal enhancers, and link these enhancers to downstream target genes. Published in Bioconductor \burl{http://bioconductor.org/packages/ELMER/}. Source code available in GitHub  \burl{https://github.com/tiagochst/ELMER}.
\end{description}

\subsection{Co-authored softwares}
\begin{description}
	\item[SpidermiR: An R/Bioconductor package for integrative network analysis with miRNA data]{Published in Bioconductor \burl{http://bioconductor.org/packages/SpidermiR/}. Source code available in GitHub \burl{https://github.com/claudiacava/SpidermiR}}.
\end{description}


\subsection{Workshops and workflows}
\begin{description}
\item[Workshop]{Integrative analysis workshop with TCGAbiolinks and ELMER. Dana Farber Cancer Institute, Boston, MA. Link to workshop: \burl{https://bioinformaticsfmrp.github.io/Bioc2017.TCGAbiolinks.ELMER/index.html}. Source code available in GitHub \burl{https://github.com/BioinformaticsFMRP/Bioc2017.TCGAbiolinks.ELMER}.}
\item[Workflow]{TCGA Workflow: Analyze cancer genomics and epigenomics data using Bioconductor packages. Available at \burl{https://www.bioconductor.org/help/workflows/TCGAWorkflow/}. Source code available in GitHub \burl{https://github.com/BioinformaticsFMRP/TCGAWorkflow}.}

\end{description}

\subsection{Conferences \& presentations}
\begin{description}
	\item[SNOLA 2016 - Update on Neuro-Oncology - Oral presentation - 04/19/2016]{"EPIGENOMIC AND TRANSCRIPTOMIC ANALYSIS OF ADULT GLIOMA REVEALS CANDIDATE DRIVER TRANSCRIPTION FACTORS INVOLVED IN GLIOMA PROGRESSION." Windsor Barra Hotel, Rio de Janeiro - Brazil}
    \item[Chromatin and Epigenetics - Poster presentation - 5/5/2017]{TCGAbiolinksGUI: A Graphical User Interface to analyze cancer genomics and epigenomics data. EMBL, Heidelberg, Germany}
	\item[Omics Seminar - Oral presentation - 06/06/2017]{Enhancer Linking by Methylation/Expression Relationship: a case study using Breast Cancer. Cedars-Sinai Medical Center, Los Angeles, California.}
	\item[Bioc2017 - Oral presentation - 07/28/2017]{Workshop: Integrative analysis workshop with TCGAbiolinks and ELMER. Dana Farber Cancer Institute, Boston, MA. Link to workshop: \burl{https://bioinformaticsfmrp.github.io/Bioc2017.TCGAbiolinks.ELMER/index.html}}
\item[From Single to Multiomics - Poster presentation - 11/13/2017]{Enhancer Linking by Methylation/Expression Relationships with the R package (ELMER) version 2. EMBL, Heidelberg, Germany}

\end{description}


%\bibliography{references}
