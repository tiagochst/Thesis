% ------------------------------------------------------------------------
% ------------------------------------------------------------------------
% ICMC: Modelo de Trabalho Acadêmico (tese de doutorado, dissertação de
% mestrado e trabalhos monográficos em geral) em conformidade com 
% ABNT NBR 14724:2011: Informação e documentação - Trabalhos acadêmicos -
% Apresentação
% ------------------------------------------------------------------------
% ------------------------------------------------------------------------

% Opções: 
%   Qualificação         = qualificacao 
%   Curso                = doutorado/mestrado
%   Situação do trabalho = pre-defesa/pos-defesa (exceto para qualificação)
% -- opções do pacote babel --
% Idioma padrão = brazil
	%spanish,			% idioma adicional para hifenização
	%english,			% idioma adicional para hifenização
	%brazil				% o último idioma é o principal do documento
\documentclass[doutorado, spanish, brazil, english]{packages/icmc}
% ---
% Pacotes Opcionais
% ---
\usepackage{rotating}           % Usado para rotacionar o texto
\usepackage[all,knot,arc,import,poly]{xy}   % Pacote para desenhos gráficos
% Este pacote pode conflitar com outros pacotes gráficos como o ``pictex''
% Então é necessário usar apenas um dos pacotes conflitantes

\newcommand{\VerbL}{0.52\textwidth}
\newcommand{\LatL}{0.42\textwidth}


% ---
% Informações de dados para CAPA e FOLHA DE ROSTO
% ---
\titulo{Development and enhancement of bioinformatic tools to integrate and understand aberrant epigenomic and genomic changes associated with cancer}
\autor[Silva, T. C.]{Tiago Chedraoui Silva}
\orientador[Orientador]{Prof. Dr.}{Houtan Noushmehr}
%\coorientador{Prof. Dr.}{Fulano de Tal}
\curso{ONCO}
\data{5}{8}{2017} % Data do depósito
% ---


% ---
% RESUMOS
% ---

% Resumo em português
% conter no máximo 500 palavras
\textoresumo{
Cancer, which is one of the major causes of mortality worldwide, is a complex disease orchestrated by aberrant genomic and epigenomic changes that can modify gene regulatory circuits and cellular identity. Emerging evidence obtained through high-throughput genomic data deposited within the public TCGA international consortium suggests that one in ten cancer patients would be more accurately classified by molecular taxonomy versus histology. Therefore, we have hypothesized that the establishment of genome-wide maps of the de novo DNA binding motifs localization coupled with differentially methylated regions and gene expression changes might help to characterize and exploit cancer phenotypes at the molecular level.

Technological advances and public databases like The Cancer Genome Atlas (TCGA), The Encyclopedia of DNA Elements (ENCODE), and The NIH Roadmap Epigenomics Mapping Consortium (roadmap) have provided unprecedented opportunities to interrogate the epigenome of cultured cancer cell lines as well as normal and tumor tissues with high resolution. Markedly however, biological information is stored in different formats and there is no current tool to integrate the data, highlighting an urgent need to develop bioinformatic tools and/or computational softwares to overcome this challenge. In this context, the main purpose of this study is the development of biOMICs, a package coded in the GNU GPL (General Public License) R programming language that will be submitted to the larger open-source Bioconductor community project. Also, we will help our collaborators improve of the R/Bioconductor ELMER package that identifies regulatory enhancers using gene expression, DNA methylation data and motif analysis.

Our expectation is that biOMICs can effectively automate search, retrieve, and analyze the vast (epi)genomic data currently available from TCGA, ENCODE, and Roadmap, and integrate genomics and epigenomics features with researchers own high-throughput data. Furthermore, we will also navigate through these data manually in order to validate the capacity of biOMICs in discovering epigenomic signatures able to redefine subtypes of cancer. Finally, we will use biOMICs and ELMER to investigate the molecular differences between two subgroups of gliomas, one of the most aggressive primary brain cancer, recently discovered by our laboratory.}{biOMICs; ELMER; Bioconductor; cancer; epigenomics; transcription factor binding sites; subgroups of gliomas.}

% ---
% resumo em inglês
% ---
\textoresumo[english]{
Cancer, which is one of the major causes of mortality worldwide, is a complex disease orchestrated by aberrant genomic and epigenomic changes that can modify gene regulatory circuits and cellular identity. Emerging evidence obtained through high-throughput genomic data deposited within the public TCGA international consortium suggests that one in ten cancer patients would be more accurately classified by molecular taxonomy versus histology. Therefore, we have hypothesized that the establishment of genome-wide maps of the de novo DNA binding motifs localization coupled with differentially methylated regions and gene expression changes might help to characterize and exploit cancer phenotypes at the molecular level.

Technological advances and public databases like The Cancer Genome Atlas (TCGA), The Encyclopedia of DNA Elements (ENCODE), and The NIH Roadmap Epigenomics Mapping Consortium (roadmap) have provided unprecedented opportunities to interrogate the epigenome of cultured cancer cell lines as well as normal and tumor tissues with high resolution. Markedly however, biological information is stored in different formats and there is no current tool to integrate the data, highlighting an urgent need to develop bioinformatic tools and/or computational softwares to overcome this challenge. In this context, the main purpose of this study is the development of biOMICs, a package coded in the GNU GPL (General Public License) R programming language that will be submitted to the larger open-source Bioconductor community project. Also, we will help our collaborators improve of the R/Bioconductor ELMER package that identifies regulatory enhancers using gene expression, DNA methylation data and motif analysis.

Our expectation is that biOMICs can effectively automate search, retrieve, and analyze the vast (epi)genomic data currently available from TCGA, ENCODE, and Roadmap, and integrate genomics and epigenomics features with researchers own high-throughput data. Furthermore, we will also navigate through these data manually in order to validate the capacity of biOMICs in discovering epigenomic signatures able to redefine subtypes of cancer. Finally, we will use biOMICs and ELMER to investigate the molecular differences between two subgroups of gliomas, one of the most aggressive primary brain cancer, recently discovered by our laboratory.}{biOMICs; ELMER; Bioconductor; cancer; epigenomics; transcription factor binding sites; subgroups of gliomas.}

% ---
% Configurações de aparência do PDF final
% ---
% alterando o aspecto da cor azul
\definecolor{blue}{RGB}{41,5,195}

% informações do PDF
\makeatletter
\hypersetup{
     	pagebackref=true,
		pdftitle={\@title}, 
		pdfauthor={\@author},
    	pdfsubject={\imprimirpreambulo},
	    pdfcreator={LaTeX with abnTeX2/ICMC-USP},
		pdfkeywords={\palavraschave}, 
		colorlinks=true,       		% false: boxed links; true: colored links
    	linkcolor=blue,          	% color of internal links
    	citecolor=blue,        		% color of links to bibliography
    	filecolor=magenta,      	% color of file links
		urlcolor=blue,
		bookmarksdepth=4
}
\makeatother
% --- 

% ----------------------------------------------------------
% ELEMENTOS PRÉ-TEXTUAIS
% ----------------------------------------------------------

% Inserir a ficha catalográfica
%\incluifichacatalografica*{tex/fichaCatalografica.pdf}
\incluifichacatalografica{634} % Código Cutter: número atribuído ao sobrenome do autor. Para obtê-lo, consulte a tabela Cutter Sanborn (em http://www.davignon.qc.ca/cutter1.html), procure pelo sobrenome ou forma mais próxima ao sobrenome completo e coloque o número indicado como parâmetro.
\incluifolhadeaprovacao*{634}

%\textofolha*{tex/pre-textual/folha-aprovacao}
% DEDICATÓRIA / AGRADECIMENTO / EPÍGRAFE
\textodedicatoria*{tex/pre-textual/dedicatoria}
\textoagradecimentos*{tex/pre-textual/agradecimentos}
\textoepigrafe*{tex/pre-textual/epigrafe}

% Inclui a lista de figuras
\incluilistadefiguras

% Inclui a lista de tabelas
\incluilistadetabelas

% Inclui a lista de quadros
\incluilistadequadros

% Inclui a lista de algoritmos
\incluilistadealgoritmos

% Inclui a lista de códigos
\incluilistadecodigos

% Inclui a lista de siglas e abreviaturas
\incluilistadesiglas

% Inclui a lista de símbolos
\incluilistadesimbolos

% ----
% Início do documento
% ----
\begin{document}
% ----------------------------------------------------------
% ELEMENTOS TEXTUAIS
% ----------------------------------------------------------
\textual

\chapter{Introdução}
\label{chapter:introducao}

\newcommand{\comando}[1]{\textbf{$\backslash$#1}}


\section{Objectives}

\section{Motivations}

\section{Organization of the Remainder of the Document}

This paper is organized in chapters as follows. In Chapter 2, we introduce
the fundamental concepts required to the development 
of this project. With this respect, we review the basic definitions genetics and epigenetics area. After presenting the theoretical concepts required to understand the work, we present the results of the work in Chapters 3, 4, and 5. Specifically, in
Chapter 3, we explore the methods and computational tools developed. 
In Chapter 4, we show the results of the use of the methods and tools presented in chapter 3 another real-world application. Enclosing this thesis, in Chapter 5, we draw the
main conclusions of this work. Moreover, the scientific contributions derived from this
project are listed, as well as possible future works. Lastly, we show the main papers
that have been published during the Doctorate period.

%Este documento explica brevemente como trabalhar com a classe \LaTeX~\textit{icmc} para confeccionar trabalhos acadêmicos seguindo as normas da \sigla{ABNT}{Associação Brasileira de Normas Técnicas} e as \aspas{\textit{Diretrizes para apresentação de dissertações e teses da USP: documento eletrônico e impresso. Parte I (ABNT)}}, publicado pelo \sigla{SIBi}{Sistema Integrado de Bibliotecas} USP. O presente manual também atende as exigências prevista no regimento do Programa de Pós-graduação em \sigla{CCMC}{Ciências da Computação e Matemática Computacional} do \sigla{ICMC}{Instituto de Ciências Matemáticas e de Computação} da \sigla{USP}{Universidade de São Paulo}.


\chapter{Instalando o abnTeX2}
\label{chapter:instalando-abntex}
\input{tex/instalando-abntex}

\chapter{Orientações gerais}
\label{chapter:orientacoes-gerais}
\input{tex/orientacoes-gerais}

\chapter{Configuração dos elementos pré-textuais}
\label{chapter:config-pre-textual}
\input{tex/config-pre-textual}

\chapter{Corpos flutuantes}
\label{chapter:corpos-flutuantes}
\input{tex/corpos-flutuantes}

\chapter{Listas}
\label{chapter:listas}
\input{tex/listas}

\chapter{Ferramentas úteis}
\label{chapter:ferramentas-uteis}
\input{tex/ferramentas-uteis}

\chapter{Citações e referências}
\label{chapter:citacoes}
\input{tex/citacoes}


% ---
% Finaliza a parte no bookmark do PDF, para que se inicie o bookmark na raiz
% ---
\bookmarksetup{startatroot}% 
% ---

% ----------------------------------------------------------
% ELEMENTOS PÓS-TEXTUAIS
% ----------------------------------------------------------
\postextual

% ----------------------------------------------------------
% Referências bibliográficas
% ----------------------------------------------------------
\bibliography{references}

% ---------------------------------------------------------------------
% GLOSSÁRIO
% ---------------------------------------------------------------------

% Arquivo que contém as definições que vão aparecer no glossário
\input{tex/glossario}
% Comando para incluir todas as definições do arquivo glossario.tex
\glsaddall
% Impressão do glossário
\printglossaries

% ----------------------------------------------------------
% Apêndices
% ----------------------------------------------------------

% ---
% Inicia os apêndices
% ---
\begin{apendicesenv}

    \chapter{Documento básico usando a classe \textit{icmc}}
    \label{chapter:documento-basico}
    \input{tex/appendix/documento-basico}
    
    \chapter{Configuração do programa JabRef}
    \label{chapter:configuracao-jabref}
    \input{tex/appendix/configuracao-jabref}

\end{apendicesenv}
% ---


% ----------------------------------------------------------
% Anexos
% ----------------------------------------------------------

% ---
% Inicia os anexos
% ---
\begin{anexosenv}

    \chapter{Páginas interessantes na Internet} 
    \label{chapter:paginas-interessantes}
    \input{tex/annex/paginas-interessantes}

\end{anexosenv}
% ---

\end{document}